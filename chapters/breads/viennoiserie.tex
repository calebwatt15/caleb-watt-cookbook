\section{Viennoiserie (Yeast Dough)}
\label{viennoiserie}
\setcounter{secnumdepth}{0}
Time: 9 hours (30 minutes prep, 7+ hours inactive rising and resting)
Serves: 1 dough packet, makes about 10-12 pastries

\begin{multicols}{2}
\subsection*{Ingredients}
\begin{itemize}
    \item \( \frac{1}{4} \) cup warm water (about 105F)
    \item \( \frac{1}{4} \) ounces dry active yeast (1 package)
    \item \( \frac{1}{8} \) teaspoon salt
    \item \( \frac{1}{2} \) teaspoon sugar
    \item 8 ounces (about 2 cups) flour that is 1 part unbleached AP flour to 2 parts whole wheat pastry flour
    \item 1 Tablespoon sugar
    \item 1 \( \frac{1}{2} \) teaspoon salt
    \item \( \frac{1}{2} \) cup milk
    \item 2 Tablespoons tasteless oil (conola works)
    \item extra flour for kneading and rolling out dough
    \item 4 ounces chilled butter (1 whole stick)
\end{itemize}

\subsection*{Hardware}
\begin{itemize}
    \item Small mixing bowl
    \item Large mixing bowl
    \item Sauce pan
    \item Large surface for rolling
    \item Heavy rolling pin
    \item Plastic with which to cover rising dough
\end{itemize}
\clearpage

\subsection*{Instructions}
\begin{enumerate}
    \item Combine \( \frac{1}{2} \) cup hot water (105F), \( \frac{1}{4} \) ounces dry active yeast, \( \frac{1}{8} \) teaspoon salt, and \( \frac{1}{2} \) teaspoon sugar in small mixing bowl.
    \item Warm \( \frac{1}{2} \) cup milk in sauce pan
    \item Combine 8 ounces of flour, 1 Tablespoon of sugar, 1 \( \frac{1}{2} \) teaspoon salt.
    \item Mix warmed milk, dry mix, and 2 Tablespoon oil, and foaming water yeast water in mixing bowl.
    \item This will be a rather sticky dough.
    \item Knead the sticky dough to start the gluten-forming process, on a large, lightly floured surface.
    \item Once the dough is fairly smooth and not too sticky (about 7 minutes of kenading), place in a clean mixing bowl and cover with plastic.
    \item Leave dough until it rises (doubles in size), about 1 \( \frac{1}{2} \) hours at 75F.
    \item Punch down risen dough in bowl, place in fridge covered to chill at least 30 minutes.
    \item Take 4 ounces chilled butter (1 whole stick) and beat down with your rolling pin to make it about \( \frac{1}{2} \) inch thick, and about as malleable as dough. This should be a flattish square of butter, about 4 inches across.
    \item Pull out the chilled dough and roll into a circle, about 10 inches across.
    \item Place flattened butter onto dough circle.
    \item Fold dough edges over butter to form a butter-dough pocket, pinch edges together to keep the pocket from leaking any butter later.
    \item Lightly flour surface again.
    \item Roll dough pocket into a 14x6 inch rectanlge, but don't roll quite to the very edge of the dough each time, the edges will remain slightly larger.
    \item Fold the dough into 3 parts, overlapping like a letter.
    \item Turn the folded dough 90 degrees. This is the end of the first "turn".
    \item Roll out the dough pocket into a 14x6 inch rectangle again.
    \item Fold the dough into three folds again, overlapping. This is the end of the second turn.
    \item Chill the folded dough in the fridge, covered, about 2 hours. This allows the gluten to rest so the dough never gets too tough.
    \item Take out the chilled dough, and roll out into another 14x6 inch rectanlge.
    \item Make three folds again and turn 90 degrees. This completes the third turn.
    \item Roll the dough out again into a 14x6 inch rectangle, make the three folds. This completes the fourth and final turn.
    \item Chill the dough, covered, again in the fridge for about 2 hours, to allow the gluten to rest and the butter to not melt.
    \item The dough can now be rolled out and used for other things, such as \nameref{croissant} or \nameref{painAuChocolat}.
    
\end{enumerate}

\subsection*{Notes}
\begin{itemize}
    \item This is based on the Croissant recipe of Julia Child and Simone Beck, as seen in Mastering the Art of French Cooking, Volume 2, page 96 and on "The French Chef", episode 1. 
    \begin{itemize}
        \item Main difference is different pastry flour, and use of weights over volumetric measurements.
    \end{itemize}
    \item While the book gives a lot of detail, watching someone make laminated dough is the best way to learn. I recommend watching the episode, which can be seen here: \url{https://www.youtube.com/watch?v=uZmrvEfhfsg}.
    \item This recipe, while French in origin, is based on American ingredients.
\end{itemize}
\end{multicols}
\clearpage