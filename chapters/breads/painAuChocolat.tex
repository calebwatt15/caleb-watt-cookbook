\section{Pain Au Chocolat}
\label{painAuChocolat}
\setcounter{secnumdepth}{0}
Time: 9 hours (30 minutes prep, 7+ hours inactive rising and resting, 18 minutes baking)
Serves: 6 pastries, 6 people, depending on generosity

\begin{multicols}{2}
\subsection*{Ingredients}
\begin{itemize}
    \item 1 recipe of \nameref{viennoiserie}
    \item Flour to roll out dough
    \item Butter for cooking sheet
    \item 12 ounces dark baking chocolate, cut into bars about 3 inches long by \( \frac{1}{4} \) in wide.
    \item 1 egg
    \item 1 teaspoon water
\end{itemize}

\subsection*{Hardware}
\begin{itemize}
    \item Large surface for rolling
    \item Knife with which to cut pastry
    \item Baking sheet
\end{itemize}
\clearpage

\subsection*{Instructions}
\begin{enumerate}
    \item Make the \nameref{viennoiserie} dough if you have not yet, this will take roughly 8 hours.
    \item Take the recently chilled and turned dough from the fridge.
    \item Lightly flour your surface.
    \item Lightly butter your baking sheet.
    \item Roll out the dough into a rectangle, about 24x6 inches.
    \item Cut the rectangle in half, so you have two 12x6 inch rectanlges.
    \item Cover and chill one rectangle while working with the other.
    \item Roll the remaining rectangle into a longer rectangle, about 16x9 inches.
    \item Place three chocolate bars end-to-end in a line across the very top of the pastry.
    \item Place additional chocolate bars in a parallel line about 7-8 inches below that.
    \item Cut the rectangle directly below the second line of chocolate, so that you have a rectangle about 8x9 inches, with a chocolate line across the very top and bottom.
    \item Cut the rectangle between each chocolate bar so you have three rectangles each with a chocolate bar along the top and bottom.
    \item Role the pastry with the chocolate bar from the top to the middle.
    \item Role the pastry from the bottom to the middle, so you have two wrapped chocolate bars meeting in the middle of a mounded pastry.
    \item Place the pastry on a baking sheet, do not crowd the pastries.
    \item Repeat with each rectangle to get 6 pastries.
    \item Make an egg wash by mixing 1 egg and 1 teaspoon water in a small bowl.
    \item Once all pastries are on baking sheet, cover lightly in egg wash.
    \item Pre-heat oven to 455F.
    \item Allow pastries to rise (double in size) on the baking sheet.
    \item Place baking sheet in oven, and allow to bake until golden brown, about 18-20 minutes.
    \item Allow pastries to cool 10-15 minutes before consuming.
    
\end{enumerate}

\subsection*{Notes}
\begin{itemize}
    \item This is based on the Croissant recipe of Julia Child and Simone Beck, as seen in Mastering the Art of French Cooking, Volume 2, page 96 and on "The French Chef", episode 1. 
    \begin{itemize}
        \item Main difference is in my \nameref{viennoiserie} dough, I have different pastry flour, and use of weights over volumetric measurements.
        \item Also, of course, I have added chocolate to make pain au chocolat, rather than croissant.
    \end{itemize}
    \item This recipe, while French in origin, is based on American ingredients and ovens.
    \item While pain au chocolat are best served fresh, they can be frozen after cooling completely. Seal in an air-tight container in the freezer.
    \item To reheat, place on a lightly buttered baking sheet into a 400F pre-heated oven for about 10 minutes.
    \item in the Southern part of France, these are called "chocolatine", and people can get a bit passionate about which name you use.
    \item I often make by dough then make half \nameref{croissant} and half Pain Au Chocolat.
    \item A huge thanks to Nicolas Bidron and Nicolas Guigo for inspiring my French baking.
\end{itemize}
\end{multicols}
\clearpage