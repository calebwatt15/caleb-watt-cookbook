\section{Croissant}
\label{croissant}
\setcounter{secnumdepth}{0}
Time: 9 hours (30 minutes prep, 7+ hours inactive rising and resting, 18 minutes baking)
Serves: 12 pastries, 6-12 people, depending on generosity

\begin{multicols}{2}
\subsection*{Ingredients}
\begin{itemize}
    \item 1 recipe of \nameref{viennoiserie}
    \item Flour to roll out dough
    \item Butter for cooking sheet
    \item 1 egg
    \item 1 teaspoon water
\end{itemize}

\subsection*{Hardware}
\begin{itemize}
    \item Large surface for rolling
    \item Knife with which to cut pastry
    \item Baking sheet
\end{itemize}
\clearpage

\subsection*{Instructions}
\begin{enumerate}
    \item Make the \nameref{viennoiserie} dough if you have not yet, this will take roughly 8 hours.
    \item Take the recently chilled and turned dough from the fridge.
    \item Lightly flour your surface.
    \item Lightly butter your baking sheet.
    \item Roll out the dough into a rectangle, about 24x6 inches.
    \item Cut the rectangle in half, so you have two 12x6 inch rectanlges.
    \item Cover and chill one rectangle while working with the other.
    \item Roll the remaining rectangle into a longer rectangle, about 15x7 inches.
    \item Cut the rectangle into three rectangles, each about 5x7 inches.
    \item Take one rectangle, roll it into a square, about 6x6 inches.
    \item Cut the rectangle along the diagonal.
    \item Take one of these right triangles and flare out the two shorter corners to match.
    \item Take the two shorter corners and roll the pastry up towards the long point.
    \item Tuck the end of the long point under the now-rolled pastry and bend the ends to shape into a crescent.
    \item Place the pastry on a baking sheet, do not crowd the pastries.
    \item Repeat with each triangle and rectangle to get 12 croissants.
    \item Make an egg wash by mixing 1 egg and 1 teaspoon water in a small bowl.
    \item Once all croissants are on baking sheet, cover lightly in egg wash.
    \item Pre-heat oven to 455F.
    \item Allow croissants to rise (double in size) on the baking sheet.
    \item Place baking sheet in oven, and allow to bake until golden brown, about 18-20 minutes.
    \item Allow croissants to cool 10-15 minutes before consuming.
    
\end{enumerate}

\subsection*{Notes}
\begin{itemize}
    \item This is based on the Croissant recipe of Julia Child and Simone Beck, as seen in Mastering the Art of French Cooking, Volume 2, page 96 and on "The French Chef", episode 1. 
    \begin{itemize}
        \item Main difference is in my \nameref{viennoiserie} dough, I have different pastry flour, and use of weights over volumetric measurements.
    \end{itemize}
    \item While the book gives a lot of detail, watching someone shape the dough is the best way to learn. I recommend watching the episode, which can be seen here: \url{https://www.youtube.com/watch?v=uZmrvEfhfsg}.
    \item This recipe, while French in origin, is based on American ingredients and ovens.
    \item While croissants are best served fresh, they can be frozen after cooling completely. Seal in an air-tight container in the freezer.
    \item To reheat, place on a lightly buttered baking sheet into a 400F pre-heated oven for about 10 minutes.
    \item I often make by dough then make half \nameref{painAuChocolat} and half croissant.
    \item A huge thanks to Nicolas Bidron and Nicolas Guigo for inspiring my French baking.
\end{itemize}
\end{multicols}
\clearpage