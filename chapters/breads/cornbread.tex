\section{Cornbread}
\label{cornbread}
\setcounter{secnumdepth}{0}
Time: 35 minutes (15 minutes prep, 20 minutes baking)
Serves: 8

\begin{multicols}{2}
\subsection*{Ingredients}
\begin{itemize}
    \item 2 tablespoons oil
    \item 1 cup yellow cornmeal
    \item 1 cup all purpose flour
    \item 1 teaspoon salt
    \item 1 teaspoon baking soda
    \item 2 teaspoons baking powder
    \item 1 tablespoons sugar
    \item 2 eggs
    \item \( \frac{1}{2} \) cup milk
    \item \( \frac{1}{2} \) buttermilk
    \item \( \frac{1}{3} \) cup vegetable oil
    \item butter for serving
\end{itemize}

\subsection*{Hardware}
\begin{itemize}
    \item Cast iron skillet
    \item Large mixing bowl
    \item Small mixing bowl
\end{itemize}
\clearpage

\subsection*{Instructions}
\begin{enumerate}
    \item Place 2 tablespoons vegetable oil into cast iron skillet.
    \item Place skillet in oven and pre-heat to 400F.
    \item Combine 1 cup cornmeal, 1 cup flour, 1 teaspoon salt, 1 teaspoon baking soda, 2 teaspoons baking powder, and 1 tablespoon sugar in large mixing bowl.
    \item Beat 2 eggs in small mixing bowl.
    \item Add \( \frac{1}{2} \) cup milk, \( \frac{1}{2} \) buttermilk, and \( \frac{1}{3} \) cup vegetable oil to beaten eggs.
    \item Mix to combine wet ingredients.
    \item Add wet ingredients to dry ingredients.
    \item Stir just to combine wet and dry.
    \item Remove skillet when oven has reached 400F.
    \item Shake skillet around to coat the whole thing in hot oil.
    \item Pour batter into hot skillet.
    \item Bake for about 30 minutes, until golden-brown on top and cooked through (can be tested with a toothpick or knife at the thickest point).
    \item Slice and serve with butter.
\end{enumerate}

\subsection*{Notes}
\begin{itemize}
    \item This is Cyndy Watt's recipe, straight up. I have not changed it from my days growing up, as my mom already makes my favorite cornbread.
    \item White cornmeal could be used, but it's less corny, and leads to a slightly softer, less tasty bread.
    \item All cornbreads sit on a spectrum ranging from sweet and cakey to drier and less-sweet. I tend to say these range from a nearly cake-like texture to actual bread. My mom's is not quite as dry as some, but is definitely not really sweet or cakey.
    \item You can add corn kernels or jalapeño if you like, but I don't normally. It's a nice special treat. Corn will make this more moist, so you may need to reduce the amount of milk.
    \item While you can use a full cup of milk instead of milk and buttermilk, it's really just not the same.
\end{itemize}
\end{multicols}
\clearpage