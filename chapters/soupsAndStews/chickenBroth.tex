\section{Chicken Broth}
\label{chickenBroth}
\setcounter{secnumdepth}{0}
Time: 3 hours 5 minutes (5 minutes prep, 3 hours cooking)
Serves: 6

\begin{multicols}{2}
\subsection*{Ingredients}
\begin{itemize}
    \item 1 whole chicken, about 5 pounds
    \item 1 onion, in quarters
    \item 2 ribs of celery, cut into quarters (no white ends)
    \item 2 carrots, cut into quarters
    \item 2 sprigs thyme
    \item 2 bay leaves
    \item 10 peppercorns
    \item optional, other herbs you like. I tend to add a couple sprigs of sage or rosemary.
    \item Enough cold water to cover the ingredients, about 2 quarts
\end{itemize}

\subsection*{Hardware}
\begin{itemize}
    \item Stock pot with lid
    \item Cheesecloth
    \item Strainer or colander
\end{itemize}
\clearpage

\subsection*{Instructions}
\begin{enumerate}
    \item Place all ingredients into the pot, and cover with cold water.
    \item Place on a burner and bring to a slow boil.
    \item Simmer for about 2-3 hours, skimming the foam and fat off the top occasionally.
    \item Strain the broth through cheesecloth in a strainer.
    \item Remove chicken from bones for use in other meals.
    \item Allow broth to cool, then store in the fridge or freezer.
\end{enumerate}

\subsection*{Notes}
\begin{itemize}
    \item This is my own recipe, but everything is pretty standard.
    \item I use this as a base for many other sauces and soups.
    \item You can remove the meat, and throw the carcass back in the broth to cook for up to 10 hours or so. This will be more flavorful, but I don't feel it's worth the added time/electricity. When I do this, I usually throw the whole things in a slow cooker over night (after removing the meat, that is).
    \item This is one recipe where the French and American technique are about identical. I learned broth from my mom, so I use the English name. In French this would be called "bouillon de poulet", I believe.
    \item Interesting to note that "bouillon" is French for broth, though in America a "bouillion cube" means strictly a dehydrated cube of meat or vegetable stock, and we don't tend to use "bouillion" as a word beyond that. Not sure why we didn't just call them "Broth Cubes" or "Stock Cubes" like most of the world.
\end{itemize}
\end{multicols}
\clearpage