\section{Tomato Soup}
\label{tomatoSoup}
\setcounter{secnumdepth}{0}
Time: 35 minutes (5 minutes prep, 30 minutes cooking)
Serves: 6

\begin{multicols}{2}
\subsection*{Ingredients}
\begin{itemize}
    \item A recipe of a white \nameref{roux}
    \item \( \frac{1}{2} \) sweet onion, chopped
    \item 28 ounces crushed tomatoes (san marzano, ideally)
    \item 16 ounces (\( \frac{1}{2} \) quart) chicken broth
    \item salt and black pepper to taste (maybe \( \frac{1}{2} \) teaspoon of each)
\end{itemize}

\subsection*{Hardware}
\begin{itemize}
    \item Stock pot
    \item Ladle
    \item Blender
\end{itemize}
\clearpage

\subsection*{Instructions}
\begin{enumerate}
    \item Make the white roux in the stock pot.
    \item As soon as the white roux has cooked for it's 5 minutes and has just begun to change color, add in \( \frac{1}{2} \) sweet onion, chopped.
    \item Keep the onions in the roux moving while they cook over medium-low heat for about 5 minute, until the onions are slightly softened.
    \item Add 28 ounces crushed tomatoes to the soup.
    \item Add 16 ounces chicken broth to the soup.
    \item Stir to combine all parts.
    \item Add salt and pepper to taste.
    \item Allow soup to come to a simmer, covered, over medium heat.
    \item Allow soup to simmer for 25 minutes, stirring occasionally, over low heat, covered.
    \item Once the soup is simmered and all the flavors have blended, remove part of it to the blender.
    \item Ensure the blender is no more than \( \frac{3}{4} \) full.
    \item Blend until the onion chunks are entirely blended.
    \item Continue to do this with the soup until it is all blended.
\end{enumerate}

\subsection*{Notes}
\begin{itemize}
    \item This is my own recipe, but everything is pretty standard.
    \item Highly recommend dipping a \nameref{grilledCheeseSandwich} in this.
    \item If the soup tastes too acidic, you can counter this with a teaspoon of baking soda. Mix it in, taste, and add \( \frac{1}{4} \) teaspoon at a time after that until it testes right. Sugar and butter can also be added.
    \item San Marzano tomatoes are preferred as they are sweeter and less acidic than other tomatoes, but really any good sauce tomato will work.
\end{itemize}
\end{multicols}
\clearpage