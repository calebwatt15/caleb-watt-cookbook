\section{Chicken And Dumplings}
\label{chickenAndDumplings}
\setcounter{secnumdepth}{0}
Time: 40 minutes (15 minutes prep, 25 minutes cooking)
Serves: 8

\begin{multicols}{2}
\subsection*{Ingredients}
\begin{itemize}
    \item 1 recipe of \nameref{chickenBroth}, about 2 quarts.
    \item a double recipe of \nameref{pieCrust}
    \item 2 cups milk
    \item heaping Tablespoon of corn starch
    \item Salt and black pepper to taste
    \item 1 chicken, deboned and cooked (from broth recipe, usually), about 4 pounds
\end{itemize}

\subsection*{Hardware}
\begin{itemize}
    \item Stock pot
    \item Small mixing bowl
\end{itemize}
\clearpage

\subsection*{Instructions}
\begin{enumerate}
    \item Heat broth to a simmer (or leave it at a simmer if you've just made the broth)
    \item Roll out the double pie crust to about 25-50\% thicker than normal.
    \item Cut the pie crust into dumplings, about 2x1 inches.
    \item Place dumplings into the simmering broth until you have as many as you want in the soup.
    \item Allow dumplings to cook at a simmer for 10 minutes.
    \item Combine 2 cups milk with 1 heaping Tablespoon corn starch in a small mixing bowl.
    \item Stir milk and corn starch until perfectly smooth.
    \item Add milk to soup, stir until thickened slightly, about 10 minutes.
    \item Season with salt and pepper to taste.
    \item Add chicken into each bowl and pour soup over it.
    \item Once soup has cooled a bit, you can add the rest of the chicken to the pot to store leftovers.
\end{enumerate}

\subsection*{Notes}
\begin{itemize}
    \item This is my mom's recipe, Cynthia Watt.
    \begin{itemize}
        \item The only difference is I make my broth and pie crust per the referenced recipes, rather than just boiling chicken in water. My mom's pie crust is just flour, salt, water, and shortening. No lard or butter.
        \item Additionally, growing up my mom would only use about \( \frac{1}{3} \) of the chicken meat in the soup. \( \frac{2}{3} \) would go into any other meal, and the bones would be boiled with any remaining attached meat scraps to make more broth, which she would use to cook chicken and rice in the oven (with just the scraps on the bones). This would net us three meals for our large family with just one chicken... ate a lot of chicken growing up.
    \end{itemize}
    \item I like my chicken and dumplings moderately thick, with a lot of pepper. Just like mama makes.
\end{itemize}
\end{multicols}
\clearpage