\section{roux}
\label{roux}
\setcounter{secnumdepth}{0}
Time: 45 minutes (Cooking between 5 and 45 minutes)
Serves: 1 roux

\begin{multicols}{2}
\subsection*{Ingredients}
\begin{itemize}
    \item 4 tablespoons vegetable oil (or butter)
    \item 6 tablespoons flour (or enough so it looks right)
\end{itemize}

\subsection*{Hardware}
\begin{itemize}
    \item Skillet
    \item Whisk
\end{itemize}
\clearpage

\subsection*{Instructions}
\begin{enumerate}
    \item Place 4 tablespoons of oil in the skillet at medium-high heat.
    \item Allow oil to come to heat (but not smoking).
    \item Place about 6 tablespoons of flour into the hot fat.
    \item Whisk the roux in the skillet constantly for a couple minutes until it is just off-white.
    \item Lower heat to medium-low and stir until the roux is at the level you need (between 5 and 45 minutes).
\end{enumerate}

\subsection*{Notes}
\begin{itemize}
    \item This is a basic roux, used as the base for many cajun dishes, as well as some other sauces.
    \item Roux can vary in color, from white, blond, brown, to dark brown. A white roux is just barely cooked flour, takes about 5 minutes total. Blond roux takes about 10-15 minutes to cook. Brown roux is 20-30 minutes, and dark brown is 35-45 minutes of cooking.
    \item White roux is still a bit flour-y, good base for milky sauces and chowders.
    \item Blond roux has a lighter flavor than brown rouxs, good base for lighter seafood.
    \item I never really use brown roux, but it's between blond and dark brown. Some people find dark brown too strong or too annoying to make.
    \item Dark brown is my favorite. It has a rich, dark color, has a slight nutty taste, and adds a great depth of flavor to any dish. This is my most commonly-used roux, especially for cajun cooking.
\end{itemize}
\end{multicols}
\clearpage