\section{Chili Con Carne}
\label{chili}
\setcounter{secnumdepth}{0}
Time: 10 hours (30 minutes prep, 1 hour cooking, 8 hour inactive cooking)
Serves: 4

\begin{multicols}{2}
\subsection*{Ingredients}
\begin{itemize}
    \item 2 teaspoons oil
    \item 1 \( \frac{1}{2} \) teaspoons kosher salt
    \item 3 pounds stew beef
    \item 3 cups (24 ounces) chicken stock
    \item 3 Tablespoons masa harina
    \item \( \frac{1}{3} \) cup water
    \item 1 medium tomato, chopped
    \item 1 small red onion, chopped
    \item 1 green bell pepper, chopped
    \item 1 Tablespoon tomato paste
    \item 1 Tablespoon \nameref{chiliPowder}
    \item 1 teaspoon ground cumin
    \item Toppings (optional), such as shredded cheese, sour cream, chives, slices fresh peppers, onions, crackers, and always \nameref{cornbread}
\end{itemize}

\subsection*{Hardware}
\begin{itemize}
    \item Medium mixing bowl
    \item Large skillet
    \item Another Medium mixing bowl
    \item Small mixing bowl
    \item Slowcooker (or enameled dutch oven)
\end{itemize}
\clearpage

\subsection*{Instructions}
\begin{enumerate}
    \item Combine 2 teaspoons cooking oil and 1 \( \frac{1}{2} \) teaspoons kosher salt in a large mixing bowl.
    \item Add 3 pounds stew beef and mix all around to coat lightly with oil and salt.
    \item Heat a large skillet (or dutch oven if using one) over high heat.
    \item Brown meat about 1-2 minutes per side in 3 or 4 batches (don't crowd the meat or the steam will prevent proper browning).
    \item Place each batch in another mixing bowl while working on each new batch.
    \item When all meat is browned, add 3 cups chicken stock to the skillet and scrape the bottom to deglaze the skillet.
    \item Pour the chicken stock into the slow cooker (or leave in dutch oven if cooking that way).
    \item Combine 3 Tablespoons masa harina with \( \frac{1}{3} \) cup water to make a heavy slurry.
    \item Add in the browned meat, 1 chopped tomato, 1 chopped bell pepper, 1 chopped red onion, masa slurry, 1 Tablespoon tomato paste, 1 Tablespoon chili powder, and 1 teaspoon ground cumin.
    \item Stir to combine all ingredients.
    \item Allow to cook over low heat, covered, for about 7 hours.
    \item Remove lid and stir 2-3 times during cooking.
    \item Allow to cook on High for the last hour.
    \item If using a dutch oven, cook in the oven at 250F for about 6 hours covered, reomving to stir 2-3 times during cooking.
    \item If it has not reduced enough, consider adding more masa slurry, or cooking over slightly higher heat (masa will thicken, heat will evaporate water).
    \item Serve with warm \nameref{cornbread}, sour cream, cheese, whatever you want.
\end{enumerate}

\subsection*{Notes}
\begin{itemize}
    \item This is mostly my own recipe after quite a few experimentations and efforts to make chili.
    \item While I find beans in chili to be delicious, I plan to add a seperate recipe for chili with beans, as that is not really traditional, nor is it proper Texan.
    \item I try to keep very little tomato in this, as I've made batches that were MUCH too acidic in the past. If it becomes too acidic, try adding some a large pinch of baking soda and stirring to combine.
    \item Masa is used as a traditional thickener in southern Texas cooking, but you could replace it with corn starch or something. I really like the flavor of masa, so I try to get and use that when I make chili.
\end{itemize}
\end{multicols}
\clearpage