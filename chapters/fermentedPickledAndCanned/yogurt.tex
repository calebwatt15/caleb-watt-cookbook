\section{Yogurt}
\label{yogurt}
\setcounter{secnumdepth}{0}
Time: 9 hours (1 hour prep, 8 hours inactive culturing)
Serves: 3

\begin{multicols}{2}
\subsection*{Ingredients}
\begin{itemize}
    \item 1 Tablespoon active yogurt
    \item 32 Ounces milk
    \item pureed fruit (optional)
\end{itemize}

\subsection*{Hardware}
\begin{itemize}
    \item Pan
    \item 32 ounce jar
    \item Thermometer
    \item Cheesecloth
\end{itemize}
\clearpage

\subsection*{Instructions}
\begin{enumerate}
    \item Heat 32 ounces of milk to 180F.
    \item Stir milk in jar until it reaches about 150F.
    \item Place jar of warm milk in ice bath, and continue stirring until it reaches about 120F.
    \item Remove jar from ice bath. Try to keep milk between 110F and 115F.
    \item Mix 1 Tablespoon active yogurt into warm milk.
    \item Let milk sit at 110F to 115F for 6 to 8 hours.
    \item Strain out resulting whey.
    \item store yogurt in the fridge with a lid.
    \item Sometimes I add pureed fruit with just my bowl of yogurt, leaving the base plain to start a new batch later.
\end{enumerate}

\subsection*{Notes}
\begin{itemize}
    \item Keep 1 Tablespoon of your plain yogurt to start your next batch in a week or so.
    \item To start off, you can use storebought yogurt that has bacteria in it (check ingredients) or buy a Yogurt starter.
    \item I had a really hard time getting a place that was consistently warm enough for the cultures to multiply (above 110F) but cold enough they did not die (below 115F). I finally found that my slow cooker set to warm, with the lid slightly open, with a water bath in it and the milk in the jar in the water bath keeps the perfect temperature.
    \item Experiment with heat sources on water prior to making this, as once the bacteria is in the milk, you will not want to go above 115F.
    \item You can strain out more or less whey to change the consistency of the yogurt.
    \item Whey can have other uses, and is rich in protein.
\end{itemize}
\end{multicols}
\clearpage