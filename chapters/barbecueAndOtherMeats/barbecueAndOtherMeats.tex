\chapter{Barbecue and Other Meats}
\label{barbecueAndOtherMeats}
\setcounter{secnumdepth}{0}
\minitoc
Smoked and grilled meats deserved their own special chapter apart from Entrées, partly due to their very different preperations, and partly because these are the epitome of good food. Barbecue is my favorite food, pretty much period.

"Barbecue is meat, prepared in a very special way..." -Rhett \& Link

Barbecue is not a grill, smoker, nor is it a cookout. While often used as a verb, it is also not a verb. Barbecue is meat, made deliciously. While traditionally barbecue is meat that is smoked or grilled, and originates in either Northern Mexico or The South, I like to take the above, broad defenition. Meat, prepared in a very special way...

This means that, to me, technically fajitas, barbacoa, jerk chicken... all barbecue (including of course, more traditional barbecue, such as smoked brisket and pulled pork.)

The next thing that should be noted is that I'm from Texas. This means I prefer beef, slow smoked over oak or hickory, and covered in a dry rub. Sauce is not really necessary. Usually if I'm having people over to partake in my delicious meat, I will also prepare a sauce, but it's only to please other people. I don't really use it much myself (at least on brisket and beef ribs, pulled pork is another story entirely.) This is because in The South the only thing that MIGHT trump meat is good ol' hospitality.

One final note. Entire books can and have been written on barbecue. If you really want to produce top tier barbecue, you need more than can be placed in a two page recipe. You need to have some understanding if your heat source, wood source, smoke, evaporative cooling, bark formation, smoke absorption, spices, cuts of meat... it's really quite a list. While other books can definitely help you get started (and I highly recommend reading a full book on Barbecue, such as "Franklin Barbecue: A Meat-Smoking Manifesto" or "Smoke and Spice"), there is no substitute for getting out there on a smoker, and making delicious, smokey meat. Try to get at least a basic understanding, then just start making meat. Worse comes to worse, you have a semi-delicious not-yet-rendered piece of meat. That still sounds edible.

Get outside, chop some wood, tend a fire, and enjoy the flesh of that animal. It's life was taken for you, you should make sure it adds the maximum amount of enjoyment to yours.

I dedicate this chapter to Edgar Black Jr. of Black's barbecue, Francisco Saucedo and Brendan Lamb of La Barbecue, Ronnie Killen of Killen's barbecue, Tim McGuffin who made a brisket and shared it with me once, Aaron Franklin of Franklin Barbecue, and maybe most importantly, Leroy "Spooney" Kenter, Jr., formerly of Spooney's Bar B Que. All of whom I've eaten their barbecue, some of who I've shared laughs with, all of whom I look up to, and a couple of whom I'll miss, whether or not we met in person.

-Caleb
\clearpage

%\input{chapters/barbecueAndOtherMeats/cajunAndouille}
%\input{chapters/barbecueAndOtherMeats/bacon}
%\input{chapters/barbecueAndOtherMeats/beefFajita}
\section{Hamburgers}
\label{hamburgers}
\setcounter{secnumdepth}{0}
Time: 45 minutes (30 minutes prep, 15 minutes cooking)
Serves: 6

\begin{multicols}{2}
\subsection*{Ingredients}
\begin{itemize}
    \item 2 pounds hamburger
    \item \( \frac{1}{4} \) cup vegetable oil
    \item Salt and black pepper to taste
    \item buns
    \item any other toppings you'd like (lettuce, tomato, bacon, cheese, etc.)
\end{itemize}

\subsection*{Hardware}
\begin{itemize}
    \item Cookie sheet
    \item Grill
\end{itemize}
\clearpage

\subsection*{Instructions}
\begin{enumerate}
    \item Take your beef and split it into 6 equal parts (each about \( \frac{1}{3} \)) pound in size.)
    \item Allow patties to rest in the fridge for about 15 minutes to firm up a tad.
    \item Ensure you have a very hot fire under the grill.
    \item Form each portion into a very wide patty (it should be wider than the buns, as it will shrink a lot).
    \item Coat each patty in a very light coat of oil.
    \item Sprinkle salt and pepper on each patty, on all sides.
    \item Place each patty down on a hot part of the grill.
    \item Close the grill door and allow the patties to cook about 3-4 minutes.
    \item Open the grill and flip each patty.
    \item Close the grill and allow to cook for another 3-4 minutes.
    \item If you want to toast buns (which I recommend) lightly butter your buns and place them on the grill for about 1-2 minutes per side.
    \item While the buns are grilling, place cheese on each patty and cover the patties in a pan with foil to allow the cheese to melt.
    \item Build your burgers and enjoy.

\end{enumerate}

\subsection*{Notes}
\begin{itemize}
    \item There is nothing fancy in this recipe. Salt, pepper, and oil to allow the seasoning to stick to the patty, and keep the patty from sticking to the grill.
    \item Do not put cheese on the patty in the grill, it will drip off and land all over your wood, which can be hard to clean.
    \item For toppings, I like to have iceberg lettuce, tomato, crispy bacon, avacado, grilled mushrooms, grilled onions, mayo, and mustard. Do what you like. This is America.
\end{itemize}
\end{multicols}
\clearpage
\section{Smoked Beef Plate Ribs}
\label{smokedBeefPlateRibs}
\setcounter{secnumdepth}{0}
Time: 10+ hours (1 hour prep, 8+ hours smoking, 1 hour resting)
Serves: 4

\begin{multicols}{2}
\subsection*{Ingredients}
\begin{itemize}
    \item 1 set of beef plate ribs (usually 4 ribs, ribs number 6 through 10)
    \item 1 recipe of \nameref{beefRibs}
    \item Tobasco Sauce
    \item spritz, 1 part apple cider vinegar to 3 parts water.
\end{itemize}

\subsection*{Hardware}
\begin{itemize}
    \item Sharp boning knife
    \item Latex gloves
    \item Offset smoker
    \item Water pan
    \item Enough wood (maybe... 25 medium logs? To be safe.)
    \item spritz bottle (for spritz)
    \item Butcher paper
    \item Thermometer with meat probe (optional, but ideal.)
\end{itemize}
\clearpage

\subsection*{Instructions}
\begin{enumerate}
    \item Trim the ribs if desired. The silverskin on the bottom will never render, but I just leave it on and eat around it. The meat will pull off of it no problem, if smoked correctly. You can trim pointy bits of fat off, if they are present. There is not too thick of a fat cap, like on a \nameref{smokedBrisket}.
    \item Slather the ribs in tobasco sauce, using a brush or latex gloves.
    \item Cover all sides of the ribs in beef rub. I usually use anywhere from half to three quarters of the rub on a rack of ribs. You don't want to cake the stuff on too thickly, but maybe a little thicker than on a brisket.
    \item Let the ribs sit for about an hour to warm up a bit while you start the fire.
    \item ensure the water pan is full and placed on the grill.
    \item Start the fire up. Use coals, paper, kindling, whatever you need to start the wood burning.
    \item You want to maintain a temperature around 285F.
    \item Place the ribs, bone-side down, on the grill.
    \item Let the ribs sit for at least 3 hours, undisturbed.
    \item Spritz the ribs every 30-45 minutes for the last 4-5 hours of cook time.
    \item You can probe the ribs in a meaty part to check for doneness. The meat should be super tender. They should be done around 203F.
    \item Wrap the ribs and let them rest for 30 minutes to an hour.
\end{enumerate}

\subsection*{Notes}
\begin{itemize}
    \item Based on Aaron Franklin's recipe, as seen in Franklin Barbecue: A Meat-Smoking Manifesto.
    \item If you have trouble regulating temperature (this happens for a lot of reasons, poor fire control, bad wood, leaky smoker, etc) the cook times can go way up. Don't worry. Just spend time, learn your smoker and wood, and keep practicing. Even a bad rack of ribs can be pretty good.
    \item While some people insist on sauce for brisket, I don't know many that ask for beef ribs.
    \item I've used primarily white oak, as I've done the most smoking since moving to Washington. Most any oak, hickory, and pecan work really well.
    \item This is specifically for beef plate ribs.
\end{itemize}
\end{multicols}
\clearpage
\section{Smoked Brisket}
\label{smokedBrisket}
\setcounter{secnumdepth}{0}
Time: 12+ hours (1 hour prep, 10+ hours smoking, 1 hour resting)
Serves: 8+

\begin{multicols}{2}
\subsection*{Ingredients}
\begin{itemize}
    \item 1 whole, untrimmed brisket (both muscles, flat and point)
    \item \( \frac{1}{4} \) cup fresh cracked black pepper (16 mesh)
    \item \( \frac{1}{4} \) cup kosher salt
    \item spritz, 1 part apple cider vinegar to 3 parts water.
\end{itemize}

\subsection*{Hardware}
\begin{itemize}
    \item Sharp boning knife
    \item latex gloves
    \item Offset smoker
    \item water pan
    \item Enough wood (maybe... 30 medium logs? To be safe.)
    \item spritz bottle (for spritz)
    \item Butcher paper
    \item thermometer with meat probe (optional, but ideal.)
\end{itemize}
\clearpage

\subsection*{Instructions}
\begin{enumerate}
    \item Trim the brisket, This means getting the fat cap down to about \( \frac{1}{4} \) in thick, removing any dangling parts, gutting off some bits... it's really easier to just watch it done once or twice. I recommend Aaron Franklin's book or Youtube videos. \url{https://www.youtube.com/watch?v=VmTzdMHu5KU}
    \item Combine \( \frac{1}{4} \) cup pepper and \( \frac{1}{4} \) cup salt in a large shaker.
    \item Cover all sides of the brisket in rub. I usually use anywhere from half to three quarters of the rub on a brisket. You don't want to cake the stuff on.
    \item Let the brisket sit for about an hour to warm up a bit while you start the fire.
    \item ensure the water pan is full and placed on the grill.
    \item Start the fire up. Use coals, paper, kindling, whatever you need to start the wood burning.
    \item You want to maintain a temperature around 275F to 285F.
    \item Place the brisket, fat-cap up, with the point (fattier side) facing the fire box.
    \item Leave the brisket sitting, undisturbed and unobserved for about 3 hours.
    \item Check the brisket. It should start to develop a bit of bark and darker color.
    \item Check the brisket every 30-45 minutes, spritzing if it looks dry.
    \item As your brisket has reached the stall, somewhere between 4 and 6 hours, usually, you may consider wrapping the brisket in butcher paper. This will keep it moist, but prevent further bark formation.
    \item After about 10 hour on the smoker, the brisket should ideally start to feel really good. It should be pliable. If you have a thermometer, you are looking for the brisket to hit about 195F to 205F internal temperature, measured at the thickest part of the flat muscle (lean side). The most important thing is that the brisket is tender.
    \item Remove the brisket from the smoker, wrap it (if it wasn't wrapped before), and allow to rest, for at least one hour.

\end{enumerate}

\subsection*{Notes}
\begin{itemize}
    \item Based on Aaron Franklin's recipe, as seen in Franklin Barbecue: A Meat-Smoking Manifesto.
    \begin{itemize}
        \item You can also view Aaron Franklin's three-part YouTube series on making a brisket, which is roughly the same information (though heavily condensed). \url{https://www.youtube.com/watch?v=VmTzdMHu5KU&list=PLJXFUkVvL7g4-ic-vMvL0VYovXzAQ3EUu}
        \item I don't use a slather on brisket, though if I was going to, I do have a spritz bottle filled with Tabasco sauce.
    \end{itemize}
    \item If you have trouble regulating temperature (this happens for a lot of reasons, poor fire control, bad wood, leaky smoker, etc) the cook times can go way up. Don't worry. Just spend time, learn your smoker and wood, and keep practicing. Even a bad brisket can be pretty great.
    \item If I were to make a sauce for this, I'd make a \nameref{Basic Sweet Barbecue Sauce}.
    \item I've used primarily white oak, as I've done the most smoking since moving to Washington. Most any oak, hickory, and pecan work really well.
\end{itemize}
\end{multicols}
\clearpage
\section{Smoked Pork Ribs}
\label{smokedPorkRibs}
\setcounter{secnumdepth}{0}
Time: 7 hours 30 minutes (1 hour prep, 6+ hours smoking, 30 minutes resting)
Serves: 4

\begin{multicols}{2}
\subsection*{Ingredients}
\begin{itemize}
    \item 1 rack of pork ribs (I prefer St. Louis Style)
    \item 1 recipe of \nameref{meatRub}
    \item spritz, 1 part apple cider vinegar to 3 parts water
    \item \( \frac{1}{3} \) cup \nameref{classicBarbecueSauce}
\end{itemize}

\subsection*{Hardware}
\begin{itemize}
    \item Latex gloves
    \item Offset smoker
    \item Water pan
    \item Enough wood (maybe... 25 medium logs? To be safe.)
    \item spritz bottle (for spritz)
    \item Heavy foil
\end{itemize}
\clearpage

\subsection*{Instructions}
\begin{enumerate}
    \item If you are using a rack of spare ribs, you'll need to trim the skirt, breastbone, and excess fat. St. Louis Style are already trimmed just fine.
    \item Cover all sides of the ribs in meat rub. I usually use anywhere from half to three quarters of the rub on a rack of ribs. You don't want to cake the stuff on too thickly, but maybe a little thicker than on a brisket.
    \item Let the ribs sit for about an hour to warm up a bit while you start the fire.
    \item ensure the water pan is full and placed on the grill.
    \item Start the fire up. Use coals, paper, kindling, whatever you need to start the wood burning.
    \item You want to maintain a temperature around 270F.
    \item Place the ribs, bone-side down, on the grill.
    \item Let the ribs sit for at least 2 hours, undisturbed.
    \item Spritz the ribs and check the color.
    \item If the ribs have a good dark, red color, slather on some barbecue sauce.
    \item Spritz a large sheet of foil and slather with a little sauce.
    \item Place the rack of ribs in the foil and completely rap with foil.
    \item Continue to smoke until the ribs are properly pliable, about 3 hours.
    \item Let the ribs rest for 30 minutes before serving.
\end{enumerate}

\subsection*{Notes}
\begin{itemize}
    \item Based on Aaron Franklin's recipe, as seen in Franklin Barbecue: A Meat-Smoking Manifesto.
    \item If you have trouble regulating temperature (this happens for a lot of reasons, poor fire control, bad wood, leaky smoker, etc) the cook times can go way up. Don't worry. Just spend time, learn your smoker and wood, and keep practicing. Even a bad rack of ribs can be pretty good.
    \item Except for the sauce used while smoking, the ribs are good without any.
    \item I've used primarily white oak, as I've done the most smoking since moving to Washington. Most any oak, hickory, and pecan work really well.
    \item While you can user other cuts of pork ribs, I prefer St. Louis Style, since they usually come trimmed and ready to smoke.
\end{itemize}
\end{multicols}
\clearpage
%\input{chapters/barbecueAndOtherMeats/smokedSalmon}
%\input{chapters/barbecueAndOtherMeats/tasso}