\chapter{Barbecue and Other Meats}
\label{barbecueAndOtherMeats}
\setcounter{secnumdepth}{0}
\minitoc
Smoked and grilled meats deserved their own special chapter apart from Entrées, partly due to their very different preperations, and partly because these are the epitome of good food. Barbecue is my favorite food, pretty much period.

"Barbecue is meat, prepared in a very special way..." -Rhett \& Link

Barbecue is not a grill, smoker, nor is it a cookout. While often used as a verb, it is also not a verb. Barbecue is meat, made deliciously. While traditionally barbecue is meat that is smoked or grilled, and originates in either Northern Mexico or The South, I like to take the above, broad defenition. Meat, prepared in a very special way...

This means that, to me, technically fajitas, barbacoa, jerk chicken... all barbecue (including of course, more traditional barbecue, such as smoked brisket and pulled pork.)

The next thing that should be noted is that I'm from Texas. This means I prefer beef, slow smoked over oak or hickory, and covered in a dry rub. Sauce is not really necessary. Usually if I'm having people over to partake in my delicious meat, I will also prepare a sauce, but it's only to please other people. I don't really use it much myself (at least on brisket and beef ribs, pulled pork is another story entirely.) This is because in The South the only thing that MIGHT trump meat is good ol' hospitality.

One final note. Entire books can and have been written on barbecue. If you really want to produce top tier barbecue, you need more than can be placed in a two page recipe. You need to have some understanding if your heat source, wood source, smoke, evaporative cooling, bark formation, smoke absorption, spices, cuts of meat... it's really quite a list. While other books can definitely help you get started (and I highly recommend reading a full book on Barbecue, such as "Franklin Barbecue: A Meat-Smoking Manifesto" or "Smoke and Spice"), there is no substitute for getting out there on a smoker, and making delicious, smokey meat. Try to get at least a basic understanding, then just start making meat. Worse comes to worse, you have a semi-delicious not-yet-rendered piece of meat. That still sounds edible.

Get outside, chop some wood, tend a fire, and enjoy the flesh of that animal. It's life was taken for you, you should make sure it adds the maximum amount of enjoyment to yours.

I dedicate this chapter to Edgar Black Jr. of Black's barbecue, Francisco Saucedo and Brendan Lamb of La Barbecue, Ronnie Killen of Killen's barbecue, Tim McGuffin who made a brisket and shared it with me once, Aaron Franklin of Franklin Barbecue, and maybe most importantly, Leroy "Spooney" Kenter, Jr., formerly of Spooney's Bar B Que. All of whom I've eaten their barbecue, some of who I've shared laughs with, all of whom I look up to, and a couple of whom I'll miss, whether or not we met in person.

-Caleb
\clearpage

%\input{chapters/barbecueAndOtherMeats/cajunAndouille}
%\input{chapters/barbecueAndOtherMeats/bacon}
%\input{chapters/barbecueAndOtherMeats/beefFajita}
%\input{chapters/barbecueAndOtherMeats/smokedBeefRibs}
%\section{Smoked Brisket}
\label{smokedBrisket}
\setcounter{secnumdepth}{0}
Time: 12+ hours (1 hour prep, 10+ hours smoking, 1 hour resting)
Serves: 8+

\begin{multicols}{2}
\subsection*{Ingredients}
\begin{itemize}
    \item 1 whole, untrimmed brisket (both muscles, flat and point)
    \item \( \frac{1}{4} \) cup fresh cracked black pepper (16 mesh)
    \item \( \frac{1}{4} \) cup kosher salt
    \item spritz, 1 part apple cider vinegar to 3 parts water.
\end{itemize}

\subsection*{Hardware}
\begin{itemize}
    \item Offset smoker
    \item Enough wood (maybe... 30 medium logs? To be safe.)
    \item spritz bottle (for spritz)
    \item Butcher paper
    \item Towel
    \item thermometer with meat probe (optional, but ideal.)
\end{itemize}
\clearpage

\subsection*{Instructions}
\begin{enumerate}
    \item Trim the brisket, This means getting the fat cap down to about \( \frac{1}{4} \) in thick, removing any dangling parts, gutting off some bits... it's really easier to just watch it done once or twice. I recommend Aaron Franklin's book or Youtube videos. \url{https://www.youtube.com/watch?v=VmTzdMHu5KU}
    \item Combine \( \frac{1}{4} \) cup pepper and \( \frac{1}{4} \) cup salt in a large shaker.
    \item Cover all sides of the brisket in rub. I usually use anywhere from half to three quarters of the rub on a brisket. You don't want to cake the stuff on.
    \item Let the brisket sit for about an hour to warm up a bit while you start the fire.
    \item Start the fire up. Use coals, paper, kindling, whatever you need to start the wood burning.
    \item You want to maintain a temperature around 275F to 285F.
    \item Place the brisket, fat-cap up, with the point (fattier side) facing the fire box.
    \item Leave the brisket sitting, undisturbed and unobserved for about 3 hours.
    \item Check the brisket. It should start to develop a bit of bark and darker color.
    \item Check the brisket every 30-45 minutes, spritzing if it looks dry.
    \item As your brisket has reached the stall, somewhere between 4 and 6 hours, usually, you may consider wrapping the brisket in butcher paper. This will keep it moist, but prevent further bark formation.
    \item After about 10 hour on the smoker, the brisket should ideally start to feel really good. It should be pliable. If you have a thermometer, you are looking for the brisket to hit about 195F to 205F internal temperature, measured at the thickest part of the flat muscle (lean side). The most important thing is that the brisket is tender.
    \item Remove the brisket from the smoker, wrap it (if it wasn't wrapped before), and allow to rest, for at least one hour.

\end{enumerate}

\subsection*{Notes}
\begin{itemize}
    \item Based on Aaron Franklin's recipe, as seen in Franklin Barbecue: A Meat-Smoking Manifesto.
    \begin{itemize}
        \item You can also view Aaron Franklin's three-part YouTube series on making a brisket, which is roughly the same information (though heavily condensed). \url{https://www.youtube.com/watch?v=VmTzdMHu5KU&list=PLJXFUkVvL7g4-ic-vMvL0VYovXzAQ3EUu}
        \item I don't use a slather on brisket, though if I was going to, I do have a spritz bottle filled with Tabasco sauce.
    \end{itemize}
    \item If you have trouble regulating temperature (this happens for a lot of reasons, poor fire control, bad wood, leaky smoker, etc) the cook times can go way up. Don't worry. Just spend time, learn your smoker and wood, and keep practicing. Even a bad brisket can be pretty great.
    \item If I were to make a sauce for this, I'd make a \nameref{Basic Sweet Barbecue Sauce}.
    \item I've used primarily white oak, as I've done the most smoking since moving to Washington. Most any oak, hickory, and pecan work really well.
\end{itemize}
\end{multicols}
\clearpage
%\input{chapters/barbecueAndOtherMeats/smokedSalmon}
%\input{chapters/barbecueAndOtherMeats/tasso}