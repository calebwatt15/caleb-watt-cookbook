\section{Smoked Pork Ribs}
\label{smokedPorkRibs}
\setcounter{secnumdepth}{0}
Time: 7 hours 30 minutes (1 hour prep, 6+ hours smoking, 30 minutes resting)
Serves: 4

\begin{multicols}{2}
\subsection*{Ingredients}
\begin{itemize}
    \item 1 rack of pork ribs (I prefer St. Louis Style)
    \item 1 recipe of \nameref{meatRub}
    \item spritz, 1 part apple cider vinegar to 3 parts water
    \item \( \frac{1}{3} \) cup \nameref{classicBarbecueSauce}
\end{itemize}

\subsection*{Hardware}
\begin{itemize}
    \item Latex gloves
    \item Offset smoker
    \item Water pan
    \item Enough wood (maybe... 25 medium logs? To be safe.)
    \item spritz bottle (for spritz)
    \item Heavy foil
\end{itemize}
\clearpage

\subsection*{Instructions}
\begin{enumerate}
    \item If you are using a rack of spare ribs, you'll need to trim the skirt, breastbone, and excess fat. St. Louis Style are already trimmed just fine.
    \item Cover all sides of the ribs in meat rub. I usually use anywhere from half to three quarters of the rub on a rack of ribs. You don't want to cake the stuff on too thickly, but maybe a little thicker than on a brisket.
    \item Let the ribs sit for about an hour to warm up a bit while you start the fire.
    \item ensure the water pan is full and placed on the grill.
    \item Start the fire up. Use coals, paper, kindling, whatever you need to start the wood burning.
    \item You want to maintain a temperature around 270F.
    \item Place the ribs, bone-side down, on the grill.
    \item Let the ribs sit for at least 2 hours, undisturbed.
    \item Spritz the ribs and check the color.
    \item If the ribs have a good dark, red color, slather on some barbecue sauce.
    \item Spritz a large sheet of foil and slather with a little sauce.
    \item Place the rack of ribs in the foil and completely rap with foil.
    \item Continue to smoke until the ribs are properly pliable, about 3 hours.
    \item Let the ribs rest for 30 minutes before serving.
\end{enumerate}

\subsection*{Notes}
\begin{itemize}
    \item Based on Aaron Franklin's recipe, as seen in Franklin Barbecue: A Meat-Smoking Manifesto.
    \item If you have trouble regulating temperature (this happens for a lot of reasons, poor fire control, bad wood, leaky smoker, etc) the cook times can go way up. Don't worry. Just spend time, learn your smoker and wood, and keep practicing. Even a bad rack of ribs can be pretty good.
    \item Except for the sauce used while smoking, the ribs are good without any.
    \item I've used primarily white oak, as I've done the most smoking since moving to Washington. Most any oak, hickory, and pecan work really well.
    \item While you can user other cuts of pork ribs, I prefer St. Louis Style, since they usually come trimmed and ready to smoke.
\end{itemize}
\end{multicols}
\clearpage