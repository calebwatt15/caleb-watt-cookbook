\section{Smoked Brisket}
\label{smokedBrisket}
\setcounter{secnumdepth}{0}
Time: 12+ hours (1 hour prep, 10+ hours smoking, 1 hour resting)
Serves: 8+

\begin{multicols}{2}
\subsection*{Ingredients}
\begin{itemize}
    \item 1 whole, untrimmed brisket (both muscles, flat and point)
    \item \( \frac{1}{4} \) cup fresh cracked black pepper (16 mesh)
    \item \( \frac{1}{4} \) cup kosher salt
    \item spritz, 1 part apple cider vinegar to 3 parts water.
\end{itemize}

\subsection*{Hardware}
\begin{itemize}
    \item Sharp boning knife
    \item latex gloves
    \item Offset smoker
    \item water pan
    \item Enough wood (maybe... 30 medium logs? To be safe.)
    \item spritz bottle (for spritz)
    \item Butcher paper
    \item thermometer with meat probe (optional, but ideal.)
\end{itemize}
\clearpage

\subsection*{Instructions}
\begin{enumerate}
    \item Trim the brisket, This means getting the fat cap down to about \( \frac{1}{4} \) in thick, removing any dangling parts, gutting off some bits... it's really easier to just watch it done once or twice. I recommend Aaron Franklin's book or Youtube videos. \url{https://www.youtube.com/watch?v=VmTzdMHu5KU}
    \item Combine \( \frac{1}{4} \) cup pepper and \( \frac{1}{4} \) cup salt in a large shaker.
    \item Cover all sides of the brisket in rub. I usually use anywhere from half to three quarters of the rub on a brisket. You don't want to cake the stuff on.
    \item Let the brisket sit for about an hour to warm up a bit while you start the fire.
    \item ensure the water pan is full and placed on the grill.
    \item Start the fire up. Use coals, paper, kindling, whatever you need to start the wood burning.
    \item You want to maintain a temperature around 275F to 285F.
    \item Place the brisket, fat-cap up, with the point (fattier side) facing the fire box.
    \item Leave the brisket sitting, undisturbed and unobserved for about 3 hours.
    \item Check the brisket. It should start to develop a bit of bark and darker color.
    \item Check the brisket every 30-45 minutes, spritzing if it looks dry.
    \item As your brisket has reached the stall, somewhere between 4 and 6 hours, usually, you may consider wrapping the brisket in butcher paper. This will keep it moist, but prevent further bark formation.
    \item After about 10 hour on the smoker, the brisket should ideally start to feel really good. It should be pliable. If you have a thermometer, you are looking for the brisket to hit about 195F to 205F internal temperature, measured at the thickest part of the flat muscle (lean side). The most important thing is that the brisket is tender.
    \item Remove the brisket from the smoker, wrap it (if it wasn't wrapped before), and allow to rest, for at least one hour.

\end{enumerate}

\subsection*{Notes}
\begin{itemize}
    \item Based on Aaron Franklin's recipe, as seen in Franklin Barbecue: A Meat-Smoking Manifesto.
    \begin{itemize}
        \item You can also view Aaron Franklin's three-part YouTube series on making a brisket, which is roughly the same information (though heavily condensed). \url{https://www.youtube.com/watch?v=VmTzdMHu5KU&list=PLJXFUkVvL7g4-ic-vMvL0VYovXzAQ3EUu}
        \item I don't use a slather on brisket, though if I was going to, I do have a spritz bottle filled with Tabasco sauce.
    \end{itemize}
    \item If you have trouble regulating temperature (this happens for a lot of reasons, poor fire control, bad wood, leaky smoker, etc) the cook times can go way up. Don't worry. Just spend time, learn your smoker and wood, and keep practicing. Even a bad brisket can be pretty great.
    \item If I were to make a sauce for this, I'd make a \nameref{Basic Sweet Barbecue Sauce}.
    \item I've used primarily white oak, as I've done the most smoking since moving to Washington. Most any oak, hickory, and pecan work really well.
\end{itemize}
\end{multicols}
\clearpage