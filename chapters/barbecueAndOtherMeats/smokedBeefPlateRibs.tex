\section{Smoked Beef Plate Ribs}
\label{smokedBeefPlateRibs}
\setcounter{secnumdepth}{0}
Time: 10+ hours (1 hour prep, 8+ hours smoking, 1 hour resting)
Serves: 4

\begin{multicols}{2}
\subsection*{Ingredients}
\begin{itemize}
    \item 1 set of beef plate ribs (usually 4 ribs, ribs number 6 through 10)
    \item \( \frac{1}{4} \) cup fresh cracked black pepper (16 mesh)
    \item \( \frac{1}{4} \) cup kosher salt
    \item Tobasco Sauce (in a spritz bottle, ideally)
    \item spritz, 1 part apple cider vinegar to 3 parts water.
\end{itemize}

\subsection*{Hardware}
\begin{itemize}
    \item Sharp boning knife
    \item latex gloves
    \item Offset smoker
    \item water pan
    \item Enough wood (maybe... 25 medium logs? To be safe.)
    \item spritz bottle (for spritz)
    \item spritx bottle for slather
    \item Butcher paper
    \item thermometer with meat probe (optional, but ideal.)
\end{itemize}
\clearpage

\subsection*{Instructions}
\begin{enumerate}
    \item Trim the ribs if desired. The silverskin on the bottom will never render, but I just leave it on and eat around it. The meat will pull of it no problem, if smoked correctly. You can trim pointy bits of fat off, if they are present. There is not too thick of a fat cap, like on a \nameref{smokedBrisket}
    \item Combine \( \frac{1}{4} \) cup pepper and \( \frac{1}{4} \) cup salt in a large shaker.
    \item Spritz the Tobasco Sauce on (or spread with latex gloves if you don't have a spritz bottle.)
    \item Cover all sides of the ribs in rub. I usually use anywhere from half to three quarters of the rub on a rack of ribs. You don't want to cake the stuff on too thickly, but maybe a little thicker than on a brisket.
    \item Let the ribs sit for about an hour to warm up a bit while you start the fire.
    \item ensure the water pan is full and placed on the grill.
    \item Start the fire up. Use coals, paper, kindling, whatever you need to start the wood burning.
    \item You want to maintain a temperature around 285F.
    \item Place the ribs, bone-side down, on the grill.
    \item Let the ribs sit for at least 3 hours, undisturbed.
    \item Spritz the ribs every 30-45 minutes for the last 4-5 hours of cook time.
    \item You can probe the ribs in a meaty part to check for doneness. The meat should be super tender. They should be done around 203F.
    \item Wrap the ribs and let them rest for 30 minutes to an hour.

\end{enumerate}

\subsection*{Notes}
\begin{itemize}
    \item Based on Aaron Franklin's recipe, as seen in Franklin Barbecue: A Meat-Smoking Manifesto.
    \item If you have trouble regulating temperature (this happens for a lot of reasons, poor fire control, bad wood, leaky smoker, etc) the cook times can go way up. Don't worry. Just spend time, learn your smoker and wood, and keep practicing. Even a bad rack of ribs can be pretty good.
    \item While some people insist on sauce for brisket, I don't know many that ask for beef ribs.
    \item I've used primarily white oak, as I've done the most smoking since moving to Washington. Most any oak, hickory, and pecan work really well.
    \item This is specifically for beef plate ribs.
\end{itemize}
\end{multicols}
\clearpage