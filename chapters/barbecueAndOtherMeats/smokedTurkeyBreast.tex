\section{Smoked Turkey Breast}
\label{smokedTurkeyBreast}
\setcounter{secnumdepth}{0}
Time: 6 hours 30 minutes (1 hour prep, 5+ hours smoking, 30 minutes resting)
Serves: 6

\begin{multicols}{2}
\subsection*{Ingredients}
\begin{itemize}
    \item 1 pair of bone-in turkey breasts, skin removed
    \item 1 recipe of \nameref{meatRub}
    \item 1 cup butter
\end{itemize}

\subsection*{Hardware}
\begin{itemize}
    \item Latex gloves
    \item Offset smoker
    \item Water pan
    \item Enough wood (maybe... 20 medium logs? To be safe.)
    \item Heavy foil
\end{itemize}
\clearpage

\subsection*{Instructions}
\begin{enumerate}
    \item Cover all sides of the turkey breasts in meat rub. I usually use anywhere from half to three quarters of the rub on a pair of breasts. You don't want to cake the stuff on too thickly, but maybe a little thicker than on a brisket.
    \item Let the breasts sit for about an hour to warm up a bit while you start the fire.
    \item ensure the water pan is full and placed on the grill.
    \item Start the fire up. Use coals, paper, kindling, whatever you need to start the wood burning.
    \item You want to maintain a temperature around 270F.
    \item Place the breasts, meat-side up, on the grill.
    \item Let the breasts sit for at least 2-2 hours, undisturbed, until they have a nice golden-brown color.
    \item Place \( \frac{1}{2} \) of the butter on each breast.
    \item Wrap the breasts in heavy foil.
    \item Continue to smoke until the breasts reach 160F.
    \item Let the braests rest for 30 minutes before serving.
\end{enumerate}

\subsection*{Notes}
\begin{itemize}
    \item Based on Aaron Franklin's recipe, as seen in Franklin Barbecue: A Meat-Smoking Manifesto.
    \item You can probably follow this for boneless turkey breasts, but they might cook faster. Aaron Franklin's book covers boneless breasts, but those are super hard to find if it's not Thanksgiving week.
    \item If you have trouble regulating temperature (this happens for a lot of reasons, poor fire control, bad wood, leaky smoker, etc) the cook times can go way up. Don't worry. Just spend time, learn your smoker and wood, and keep practicing. Even a bad rack of ribs can be pretty good.
    \item I've used primarily white oak, as I've done the most smoking since moving to Washington. Most any oak, hickory, and pecan work really well.
\end{itemize}
\end{multicols}
\clearpage