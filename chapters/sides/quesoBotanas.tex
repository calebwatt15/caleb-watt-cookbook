\section{Queso Botanas}
\label{quesoBotanas}
\setcounter{secnumdepth}{0}
Time: 30 minutes (10 minutes prep, 20 minutes cooking)
Serves: Queso for 2 botanas

\begin{multicols}{2}
\subsection*{Ingredients}
\begin{itemize}
    \item 15 ounces milk
    \item \( \frac{1}{2} \) ounces sodium citrate
    \item 30 ounces american cheese, shredded
    \item Optionally, for queso as a dip, grilled onions and peppers.
\end{itemize}

\subsection*{Hardware}
\begin{itemize}
    \item Sauce pan
    \item Whisk
\end{itemize}
\clearpage

\subsection*{Instructions}
\begin{enumerate}
    \item Bring 30 ounces milk to a simmer in a sauce pan.
    \item Add \( \frac{1}{2} \) ounces sodium citrate and whisk to combine.
    \item Add shredded american cheese, , a small handful at a time.
    \item Whisk the freshly-added cheese heavily and allow it to fully incorporate into the liquid before adding more.
    \item Continue the above 2 steps until all of the cheese is incorporated in the sauce.
    \item When hot this cheese is liquid, if it cools it becomes almost the consistency of Velveeta. You want something in between for botanas. Liquid is perfect for queso.
    \item If you want to use this queso as a dip, which is delicious, and my goto queso dip, then I recommend grilling some pieces of onion and bell peppers, then combining with the liquid cheese.
\end{enumerate}

\subsection*{Notes}
\begin{itemize}
    \item This technique is the same as the \nameref{macAndCheeseEmulsified}, as I learned from Modernist Cuisine.
    \begin{itemize}
        \item The main difference is that this cheese is MUCH thicker than that, and I only use American cheese for botanas.
    \end{itemize}
    \item As the name implies, this is a queso I primarily use as a component of my \nameref{botanas}.
    \item If you want queso for dip, and not for botanas, then using a cheese other than American makes sense. Maybe something like queso oaxaca (though American also works for this). I often make this recipe, then use half on botanas, and half as dip.
    \item In the Rio Grande Valley, where botanas originate and are plentiful, nearly every single restuarant uses "Land O' Lakes' American Extra Melt". This is to goto cheese for good quesos and botanas. The issue is that Land O' Lakes does not sell this to indivuduals. They only sell it in crates with 6 boxes of 5 pound blocks each. I've never had the oppurtunity to need 30 pounds of cheese at once, so I set out to re-create it from scratch. This is about as close as you can get.
    \item American Extra Melt is basically high quality American cheese, with added emuslifiers to help it stay emulsified as it heats up.
    \item Growing up I always thought "The cheese on this botana feels like the consistency of velveeta, but tastes really good." It wasn't until I called Taco Ole and Treviño's that I learned that most places are using this one brand and type of cheese, which is hard to find.
    \item Much like the Mac and Cheese, the quality of cheese put in determines the quality coming out. Please buy a good American cheese from your local cheese shop or deli. It will help immensely.
\end{itemize}
\end{multicols}
\clearpage