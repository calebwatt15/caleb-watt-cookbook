\section{Arroz a la Mexicana}
\label{arrozALaMexicana}
\setcounter{secnumdepth}{0}
Time: 35 minutes (5 minutes prep, 30 minutes cooking)
Serves: 4

\begin{multicols}{2}
\subsection*{Ingredients}
\begin{itemize}
    \item 3 Tablespoons vegetable oil
    \item 1 cup long grain white rice
    \item \( \frac{1}{2} \) teaspoon cumin powder
    \item 1 clove garlic, minced
    \item 2 ounces (about \( \frac{1}{4} \) of a) yellow onion, diched
    \item scant \( \frac{1}{2} \) cup tomato purée
    \item 2 cups chicken broth
    \item \( \frac{1}{4} \) teaspoon kosher salt
\end{itemize}

\subsection*{Hardware}
\begin{itemize}
    \item Sauce pan with lid
\end{itemize}
\clearpage

\subsection*{Instructions}
\begin{enumerate}
    \item Heat 3 Tablespoons vegetable oil in a sauce pan over medium heat.
    \item Add 1 cup long grain rice, stir and move constantly until rice is lightly toasted, about 7-9 minutes.
    \item Add 1 diced clove of garlic, 2 ounces of diced yellow onion, and \( \frac{1}{2} \) teaspoon cumin powder. Stir to combine.
    \item Cook for about 1 minute.
    \item Add scant \( \frac{1}{2} \) cup tomato purée and 2 cups chicken broth, stir to combine.
    \item Bring the rice to a boil.
    \item Cover the pan, lower heat to low, and let cook for 19 minutes.
    \item Remove from heat, lightly stir and fluff with a fork, and serve warm.
\end{enumerate}

\subsection*{Notes}
\begin{itemize}
    \item This is my own recipe, but is fairly standard.
    \item If you find this is too wet, reduce water a little, as the onion may add too much water to the dish.
    \item This is a standard side to many mexican and tex-mex dishes. Try \nameref{botanas}.
\end{itemize}
\end{multicols}
\clearpage