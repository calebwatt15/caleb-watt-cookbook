\section{Dirty Rice}
\label{dirtyRice}
\setcounter{secnumdepth}{0}
Time: 2 hours (45 minutes prep, 1 hour 15 minutes cooking)
Serves: 8

\begin{multicols}{2}
\subsection*{Ingredients}
\begin{itemize}
    \item \( \frac{1}{4} \) pound chicken gizzards, chopped
    \item \( \frac{1}{2} \) cup water
    \item 1 medium onion, chopped
    \item 2 ribs of celery, chopped
    \item 2 green bell peppers, deseeded and chopped
    \item 3 tablespoons of vegetable oil
    \item \( \frac{1}{4} \) pound of chicken livers, chopped
    \item \( \frac{1}{2} \) pound ground beef
    \item 2 tablespoons parsley, chopped
    \item 1 beef bouillon cube, dissolved in \( \frac{1}{2} \) cup hot water
    \item 1 tablespoon worcestershire sauce
    \item Salt, red pepper, and black pepper to taste
    \item 3 cups steamed white rice (standard steamed white rice recipe)
\end{itemize}

\subsection*{Hardware}
\begin{itemize}
    \item Dutch oven
\end{itemize}
\clearpage

\subsection*{Instructions}
\begin{enumerate}
    \item Simmer \( \frac{1}{4} \) pound chicken gizzards in \( \frac{1}{2} \) cup water for 20 minutes.
    \item Remove gizzards and water from dutch oven.
    \item Saute chopped onion, celery, and bell peppers in oil until soft, at least 10 minutes.
    \item Add liver, gizzards, ground beef, and parsley.
    \item Brown meat thoroughly at medium heat.
    \item Add beef bouillon liquid and spices. Cover and simmer at low heat for 30 minutes.
    \item Remove from heat, add to rice, mix well.

\end{enumerate}

\subsection*{Notes}
\begin{itemize}
    \item Based on Mrs. Landen Alexander’s recipe, from Brusly (West Baton Rouge Parish), as seen in Acadiana Profile’s Cajun Cooking: From the Kitchens of South Louisiana, Part 1, 1990.
    \begin{itemize}
        \item Main difference is the addition of simmering the gizzards first, in order to soften them up. Additionally, I changed the odd ratio of Holy Trinity from the original recipe.
    \end{itemize}
    \item Haven’t actually tried the gizzard simmering yet, need to cook this again.
\end{itemize}
\end{multicols}
\clearpage