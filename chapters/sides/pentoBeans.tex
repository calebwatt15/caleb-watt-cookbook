\section{Pento Beans}
\label{pentoBeans}
\setcounter{secnumdepth}{0}
Time: 3 hours 30 minutes (10 minutes prep, 3 hours 20 minutes cooking)
Serves: 6

\begin{multicols}{2}
\subsection*{Ingredients}
\begin{itemize}
    \item 32 ounces pento beans
    \item 16 cups water (8 cups for a quick boil, 6 for cooking)
    \item 5 ounces (about \( \frac{1}{2} \) of a) sweet onion, large chunks
    \item 2 cloves garlic, whole
    \item 2 Tablespoon salt
\end{itemize}

\subsection*{Hardware}
\begin{itemize}
    \item Large stock pot with lid
\end{itemize}
\clearpage

\subsection*{Instructions}
\begin{enumerate}
    \item Place 32 ounces pento beans and 8 cups water in a large stock pot.
    \item bring to a rapid boil, and cook for 2 minutes.
    \item Remove from heat and allow to sit, covered, for one hour.
    \item Rinse beans out.
    \item Add 6 cups water to the beans.
    \item Add 5 ounces onion, 2 cloves garlic, and 2 tablespoons salt.
    \item Allow to simmer, covered, for about 2 hours (until tender).
    \item Remove onion and garlic before serving.
\end{enumerate}

\subsection*{Notes}
\begin{itemize}
    \item This is my own recipe, but is fairly standard.
    \item These are not super flavorful on their own. I tend to turn them into \nameref{frijolesCharros} or \nameref{frijolesRefritos} (both are great sides for Mexican and Tex-Mex food).
    \item This is a really standard way to cook beans, and will work with most beans.
    \item You can soak the beans over night rather than doing the initial boil. Both ways make for good final beans.
\end{itemize}
\end{multicols}
\clearpage