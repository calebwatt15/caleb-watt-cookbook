\section{Mac and Cheese (emulsified)}
\label{macAndCheeseEmulsified}
\setcounter{secnumdepth}{0}
Time: 30 minutes (20 minutes prep, 10 minutes cooking)
Serves: 6

\begin{multicols}{2}
\subsection*{Ingredients}
\begin{itemize}
    \item 1 quart water
    \item 1 pound elbow macaroni (or shells... love me some shells)
    \item 500 grams of milk
    \item 14 grams sodium citrate
    \item 575 grams cheese, freshly grated
    \item Salt and pepper to taste
\end{itemize}

\subsection*{Hardware}
\begin{itemize}
    \item Small stock pot
    \item Stainless steel sauce pan
    \item Wire whisk
    \item Skillet (optional, for bacon)
\end{itemize}
\clearpage

\subsection*{Instructions}
\begin{enumerate}
    \item If you want to add bacon, cook it per the \nameref{crispyBacon} recipe.
    \item Bring 1 quart water to a good boil over High heat.
    \item While cooking bacon, cook the 1 pound of macaroni to just before al dente (usually about 5 minutes) in the boiling water.
    \item As the macaroni is cooking, bring the 500 grams of water to a simmer in the stainless steel sauce pan.
    \item Dissolve 14 grams of sodium citrate in the simmering water.
    \item Add 570 grams of shredded cheese, a small handful at a time.
    \item Whisk the freshly-added cheese heavily and allow it to fully incorporate into the liquid before adding more.
    \item Continue the above 2 steps until all of the cheese is incorporated in the sauce.
    \item Drain the macaroni of water, but do not rinse.
    \item Add cheese sauce to the macaroni.
    \item Season with salt and pepper.
    \item (optional steps): Steam the broccoli and stir in with the cheese.
    \item (optional step) Chop the bacon into small pieces and stir in with the cheese.

\end{enumerate}

\subsection*{Notes}
\begin{itemize}
    \item Based on the Modernist Cuisine Silky Mac and Cheese recipe: \url{http://modernistcuisine.com/recipes/silky-smooth-macaroni-and-cheese/}
    \begin{itemize}
        \item Main differences are it's scaled up to a pound of macaroni.
        \item Slightly higher ratio of cheese to milk, to get a thicker sauce.
    \end{itemize}
    \item Please note that this recipe does not scale, particularly the sodium citrate. Modernist Cuisine states that the water should scale up 93\%, sodium citrate up 4\%, cheese 100\%, and pasta 84\%, however this recipe is a little off from those numbers, so I recommend experimenting with larger batches.
    \item Cheese is an emulsion of the dairy and the oily fats. Normally when heat is applied (cheese is melted) this emulsion breaks down. This results in oily cheese which is less than ideal. This recipe is based on the idea of adding an emulsifier to help hold that emulsion even under higher heat. This allows us to create a liquid cheese (like velveeta), but it tastes like good quality cheese.
    \item While cheddar is traditional, this works well with most melty cheeses (nothing too soft or too hard), think colby jack, cheddar, or even pepper jack. Avoid soft cheeses like brie, and hard cheeses like cotija.
    \item The tastier the cheese, the tastier the end sauce.
    \item Sodium citrate is one of many emulsifiers, however it's easy to find on Amazon. Sodium Phosphate should work in theory, and is used in things like Land o' Lakes American Extra Melt, however it's not as readily available. Sodium citrate can also be found in the kosher section of the grocery store sometimes, however it is named "sour salt". Check the ingredients to make sure it is pure sodium citrate, and not citric acid, which will not work.
\end{itemize}
\end{multicols}
\clearpage