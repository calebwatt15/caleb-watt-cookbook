\section{Mashed Potatos}
\label{mashedPotatos}
\setcounter{secnumdepth}{0}
Time: 45 minutes (15 minutes prep, 30 minutes cooking)
Serves: 4

\begin{multicols}{2}
\subsection*{Ingredients}
\begin{itemize}
    \item 3 pounds potatos, peeled and chopped into \( \frac{1}{2} \) inch pieces
    \item 2 teaspoons salt
    \item 1 \( \frac{1}{2} \) teaspoons black pepper
    \item \( \frac{1}{4} \) cup butter, cut into \( \frac{1}{2} \) tablespoon sized pieces
    \item \( \frac{1}{4} \) cup sour cream
\end{itemize}

\subsection*{Hardware}
\begin{itemize}
    \item Large Pot (at least 4 quarts)
    \item Fork
\end{itemize}
\clearpage

\subsection*{Instructions}
\begin{enumerate}
    \item Start about 3 quarts of water boiling in a large pot.
    \item Place potato chunks into water and allow to boil until soft, about 15 minutes.
    \item Drain water from potatos, then return potatoes to pot.
    \item Add \( \frac{1}{4} \) cup butter into potatos.
    \item Add 2 teaspoons salt and 1 \( \frac{1}{2} \) teaspoons black pepper (more or less to taste).
    \item Add \( \frac{1}{4} \) cup sour cream to potatos.
    \item Mash potatos with a fork, stirring and mixing in all ingredients at the same time.
\end{enumerate}

\subsection*{Notes}
\begin{itemize}
    \item Recipe is based on Cyndy Watt's, who taught me a lot of things about cooking growing up.
    \item If you refuse to use a fork, a stand mixer, hand mixer, or that silly wavy wire thingy for mashing all work just fine.
    \item You can leave different amounts of chunks in the final potatos, per your preference, however for dishes such as \nameref{americanCottagePie} I prefer very smooth potatos.
    \item These are especially delicious with \nameref{countryGravy} or \nameref{brownGravy} if you'd rather.
\end{itemize}
\end{multicols}
\clearpage