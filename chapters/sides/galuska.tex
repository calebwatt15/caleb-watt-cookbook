\section{Galuska (Nokedli)}
\label{galuska}
\setcounter{secnumdepth}{0}
Time: 25 minutes (5 minutes prep, 10 minutes resting, 10 minutes cooking)
Serves: 4

\begin{multicols}{2}
\subsection*{Ingredients}
\begin{itemize}
    \item 1 egg
    \item 1 tablespoons lard
    \item \( \frac{1}{3} \) cup water
    \item 1 teaspoon salt
    \item 1 \( \frac{1}{2} \) cups flour
    \item 3 quarts water
    \item 1 tablespoons water
    \item 2 tablespoons lard
\end{itemize}

\subsection*{Hardware}
\begin{itemize}
    \item Medium mixing bowl
    \item 4 Quart stock pot
    \item Small spoon
    \item Slotted spoon
    \item skillet
\end{itemize}
\clearpage

\subsection*{Instructions}
\begin{enumerate}
    \item Mix 1 egg, 1 tablespoon lard, \( \frac{1}{3} \) cup water, and 1 teaspoon of salt.
    \item Lightly mix in 1 \( \frac{1}{2} \) cups flour (maybe 3 minutes of mixing). Do not overwork.
    \item Let the dough rest for 10 minutes.
    \item Bring 3 quarts of water and 1 Tablespoon of salt to boil.
    \item Dip small spoon into boiling water to prevent sticking.
    \item Use spoon to tear pieces of dough off, the size you want your galuskas to be.
    \item Drop each piece into the water, do this quickly, so they all come out as close as possible.
    \item When all galuskas have floated to the top remove them with a slotted spoon.
    \item Drain all water from galuskas.
    \item Heat 2 tablespoons lard in a skillet at medium heat.
    \item Lightly toss the galuskas in heated lard.
    \item You can salt them further if you like, however I commonly eat them with a salty stew or sauce, so I don't usually salt more.

\end{enumerate}

\subsection*{Notes}
\begin{itemize}
    \item The recipe is from Gearge Lang's "The Cuisine Of Hungary", 1971 edition. Page 297. I have only broken up the amounts of ingredients and clarified and added cook times and temperatures to more steps.
    \item These are also called "nokedli" in Hungarian cuisine, and they are also pretty much "Spätzle" that is seen commonly in German cuisine.
    \item The size is up to you, though I tend to make them about a half teaspoon in size (each varies a little).
    \item These are commonly eaten with \nameref{csirkePaprikas} or \nameref{goulyas}.
\end{itemize}
\end{multicols}
\clearpage