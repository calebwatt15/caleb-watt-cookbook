\section{American Cottage Pie}
\label{americanCottagePie}
\setcounter{secnumdepth}{0}
Time: 2 hours (45 minutes prep, 1 hour and 15 minutes cooking)
Serves: 8

\begin{multicols}{2}
\subsection*{Ingredients}
\begin{itemize}
    \item 1 pound of fresh or frozen green beans
    \item 1 \( \frac{1}{2} \) pounds ground beef
    \item about 1 \( \frac{1}{2} \) teaspoons salt for meat, to taste
    \item Pepper to taste
    \item 1 recipe of \nameref{creamOfMushroomSoup}
    \item 2 pounds mashed potatos (see \nameref{mashedPotatos} recipe)
    \item 2 \( \frac{1}{2} \) cups cheese, grated
\end{itemize}

\subsection*{Hardware}
\begin{itemize}
    \item 2 quart pot to boil vegetables in
    \item Dutch Oven
    \item 9x13 casserole dish
\end{itemize}
\clearpage

\subsection*{Instructions}
\begin{enumerate}
    \item If you have not made the mashed potatos, do that first.
    \item Boil or steam 1 pound of green beans to the point where they are just a bit underdone.
    \item Remove green beans from heat, drain water.
    \item Brown the meat in a dutch oven, adding salt and pepper to taste.
    \item Remove the meat from the dutch oven.
    \item Drain grease from meat.
    \item If you have not yet, make the recipe of \nameref{creamOfMushroomSoup}.
    \item When the cream is almost to the desired thickness, add in ground beef and green beans. Stir to combine.
    \item Remove soup and meat from heat.
    \item Place the soup and meat mixture into the bottom of a 9x13 casserole dish.
    \item Place the mashed potatos in a flat layer on top of the meat.
    \item Sprinkle the grated cheese to cover the potatos.
    \item Throw the entire dish in the oven, set to 350F.
    \item When the cheese is melted and the rest of the dish is warm, remove and serve.
\end{enumerate}

\subsection*{Notes}
\begin{itemize}
    \item Recipe is based on Cyndy Watt's recipe, however growing up we used canned cream of mushroom and green beans. Obviously we did NOT use instant mashed potatos, as we were not barbarians.
    \item The core difference between a cottage pie and a shepherd's pie is the use of ground beef instead of ground lamb. This recipe works fine as a shepherd's pie with lamb instead. Be careful not to overcook the lamb, as it dries out fairly easily.
    \item I call this American cottage pie as it is covered in cheese, while a \nameref{traditionalShepherdsPie} is a meat and gravy layer under mashed potatos that are crisped slightly, no cheese. This is basically how my mom made it growing up in Texas.
\end{itemize}
\end{multicols}
\clearpage