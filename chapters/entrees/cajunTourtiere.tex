\section{Cajun Tourtière}
\label{cajunTourtiere}
\setcounter{secnumdepth}{0}
Time: 2 hours (30 minutes prep, 45 minutes cooking, 45 minutes baking)
Serves: 6-8

\begin{multicols}{2}
\subsection*{Ingredients}
\begin{itemize}
    \item 2 Pounds ground pork
    \item \( \frac{1}{2} \) cup chopped onion
    \item \( \frac{1}{2} \) cup chopped celery
    \item \( \frac{1}{2} \) cup chopped green bell pepper
    \item 1 clove garlic, chopped
    \item \( \frac{1}{4} \) cup chopped parsley
    \item 1 Teaspoon salt
    \item \( \frac{1}{2} \) Teaspoon black pepper
    \item \( \frac{1}{2} \) Teaspoon cayenne pepper
    \item \( \frac{1}{4} \) Teaspoon crushed marjoram leaf
    \item \( \frac{1}{8} \) Teaspoon ground cloves
    \item \( \frac{1}{8} \) Teaspoon ground cinnamon
    \item 2 Tablespoons flour
    \item 2 beef bouillon cubes
    \item 1 cup hot water
    \item 2 pie crusts (see \nameref{pieCrust} recipe)
    \item 1 egg
\end{itemize}

\subsection*{Hardware}
\begin{itemize}
    \item Dutch oven
    \item Pie dish
\end{itemize}
\clearpage

\subsection*{Instructions}
\begin{enumerate}
    \item Saute 2 pounds ground pork, \( \frac{1}{2} \) cup chopped onion, \( \frac{1}{2} \) cup chopped celery, \( \frac{1}{2} \) cup chopped green bell pepper, and 1 clove garlic until pork is browned and vegetables are soft, about 10 minutes in a dutch oven.
    \item Stir in \( \frac{1}{4} \) cup parsley, 1 teaspoon salt, \( \frac{1}{2} \) teaspoon black pepper, \( \frac{1}{2} \) teaspoon cayenne pepper, \( \frac{1}{4} \) teaspoon marjoram, \( \frac{1}{8} \) Teaspoon ground cloves, and \( \frac{1}{8} \) Teaspoon ground cinnamon.
    \item Cover and simmer on low heat for about 30 minutes.
    \item Drain excess fat from the dutch oven.
    \item Stir flour into the meat mixture.
    \item Dissolve 2 beef bouillon cubes into 1 cup hot water.
    \item Add beef bouillon into the meat mixture.
    \item Bring to medium-high heat, and allow mixture to begin to boil.
    \item Allow mixture to boil for 1 minute while stirring constantly.
    \item Remove from heat and set aside to cool completely.
    \item If you have already made pie crusts, place one in the bottom of a pie pan.
    \item Mound cooled meat mixture into the pie pan, and cover with second crust.
    \item Seal edges by fluting the crusts together. No need to cut slits into the crust.
    \item Brush top crust with egg.
    \item Bake at 400F for about 45 minutes, until the crust is golden-brown.
\end{enumerate}

\subsection*{Notes}
\begin{itemize}
    \item Based on Mrs. Hazel Gourgues' recipe, from Hahnville (St. Charles Parish), as seen in Acadiana Profile’s Cajun Cooking: From the Kitchens of South Louisiana, Part 1, 1990.
    \begin{itemize}
        \item Main differences are less celery, addition of green bell peppers, removal of mace, addition of cayenne pepper, use of homemade crusts (that's a given).
    \end{itemize}
    \item While the traiditonal Tourtière is a Quebecois Christmas staple, this version is closer to the Acadian Tourtière. The spices are still very aromatic, however they have a cajun flair.
    \item Also works well with wild game, duck, chicken, or seafood.
\end{itemize}
\end{multicols}
\clearpage