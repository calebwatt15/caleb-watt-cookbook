\section{Boeuf Bourguignon}
\label{boeufBourguignon}
\setcounter{secnumdepth}{0}
Time: 4 hours (30 minutes prep, 1 hours baking)
Serves: 8

\begin{multicols}{2}
\subsection*{Ingredients}
\begin{itemize}
    \item 6 cups of water
    \item 2 slices of thick-cut bacon, cut into lardons (strips, 1 inch by \( \frac{1}{4} \) inch)
    \item 1 Tablespoon olive oil
    \item A 3+ pound chuck roast, cut into 2 inch squares
    \item 1 carrot, sliced
    \item \( \frac{1}{2} \) onion, sliced
    \item 1 teaspoon salt
    \item \( \frac{1}{2} \) teaspoon pepper
    \item 2 Tablespoons flour
    \item 1 bottle of red wine. Something full-bodied, young and French. (Beaujolais works well)
    \item 2-3 cups beef stock
    \item 1 Tablespoon tomato paste
    \item 2 cloves garlic, minced
    \item \( \frac{1}{2} \) teaspoon thyme (fresh or dried)
    \item 1 bay leaf, crumpled
    \item 1 recipe of \nameref{oignonsGlacesABrun}
    \item 1 recipe of \nameref{champignonsSautesAuBeurre}
\end{itemize}

\subsection*{Hardware}
\begin{itemize}
    \item Oven-safe stock pot
    \item Colander
    \item Sauce pan
\end{itemize}
\clearpage

\subsection*{Instructions}
\begin{enumerate}
    \item Place 6 cups of water in stock pot and bring to simmer over medium.
    \item Place lardons (bacon strips) in water, simmer for 10 minutes.
    \item Drain lardons and pat dry.
    \item Throw water out.
    \item Heat 1 tablespoon olive oil in stock pot.
    \item Sauté lardons in oil over medium heat, 2-3 minutes (lightly browned, but not too crispy)
    \item Remove bacon from stock pot, leave the fat and oil.
    \item Increase heat to High, allow fat to get very hot, but not quite smoking.
    \item Ensure the beef is dry, pat dry if necessary.
    \item brown the beef in batches. Allow all sides to brown, and do not crowd the beef.
    \item Add beef to the bacon.
    \item Brown the sliced carrot and sliced onion, about 10 minutes.
    \item Remove from heat, pour out the fat.
    \item Put lardons and beef back in the stock pot.
    \item Toss the beef with 1 teaspoon salt and \( \frac{1}{2} \) teaspoon pepper.
    \item Sprinkle on 2 tablespoons flour, toss to coat the beef.
    \item Set the stock pot, uncovered in a 450F oven, bottom third, for about 4 minutes.
    \item Stir the meat around and allow to cook in the over for another 4 minutes.
    \item Remove the stock pot from the oven, reduce the heat to 350F.
    \item Stir in one bottle of wine and 2 cups beef stock (really just enough stock to cover the meat, usually 2 cups).
    \item Add 1 tablespoon tomato paste, 2 cloves minced garlic, \( \frac{1}{2} \) teaspoon thyme, and 1 crumpled bay leaf.
    \item Stir to combine.
    \item Bring the stew to a simmer on the stove top.
    \item Cover the stew and place in the oven to stew for about 3 hours, until the meat is easily pierced with a fork.
    \item If you have not pre-made the oignons and champignons (onions and mushrooms), this is a good time.
    \item Once the meat is tender pour the stew through a colander into a sauce pan.
    \item Wash out the stock pot real quick.
    \item Return the solid part of the stew to the stockpot.
    \item You should have at least 2-3 cups of sauce, just thick enough to coat a spoon.
    \item If the sauce is not thick enough to coat a spoon, reduce it over medium-high heat until it is.
    \item Place the oignons and champignons over the stew.
    \item Skim fat off the sauce (though leave a touch for delicious flavor).
    \item Pour the sauce over the stew.
\end{enumerate}

\subsection*{Notes}
\begin{itemize}
    \item This is based on the recipe of Julia Child, Simone Beck, and Louisette Bertholle, as seen in Mastering the Art of French Cooking, Volume 1, page 315.
    \begin{itemize}
        \item Main differences are slightly more wine, less bacon, no bacon rind (too hard to find in the states these days), and slightly different spice amounts (salt and pepper).
    \end{itemize}
    \item Translates to "Beef bourguignon", this is a stew from the Burgandy region of France. Originally peasant food.
    \item While this is technically a stew, I have placed it in the entreés as I tend to eat it as a main meal, and on a plate, served over potatoes.
    \item Traditionally served with \nameref{frenchBoiledPotatoes}.
    \item I often serve it on top of \nameref{mashedPotatoes} or even on top of \nameref{bakedPotatoes}.
\end{itemize}
\end{multicols}
\clearpage