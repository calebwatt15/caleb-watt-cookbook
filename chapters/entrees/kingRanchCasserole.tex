\section{King Ranch Casserole}
\label{kingRanchCasserole}
\setcounter{secnumdepth}{0}
Time: 2 hours (45 minutes prep, 1 hour 15 minutes cooking)
Serves: 6

\begin{multicols}{2}
\subsection*{Ingredients}
\begin{itemize}
    \item 1 recipe \nameref{creamOfMushroomSoup}
    \item 1,150 grams chicken (I tend to use deboned, deskinned thighs), chopped
    \item 2 teaspoons salt
    \item 1 teaspoon black pepper
    \item Vegetable oil for pan frying
    \item 1 small (about 100 grams) sweet onion, chopped
    \item 2 (about 350 grams) bell peppers, chopped
    \item 120 grams (\( \frac{1}{2} \) cup) \nameref{chickenBroth} (You can use the chicken from making broth, but I tend to brown new chicken instead)
    \item 2 teaspoons ground cumin
    \item 2 teaspoons \nameref{chiliPowder} (Add 50\% more if using store bought)
    \item 1 Tablespoon kosher salt
    \item \( \frac{1}{2} \) teaspoons dried oregano
    \item 283 grams tomatoes, diced (traditionally uses Ro*Tel with green chiles)
    \item 1 teaspoon corn starch disolved into 1 Tablespoon warm water (optional, if sauce is too thin)
    \item Oil, butter, or no-stick spray
    \item 16 ounces melting cheese, shredded (colby jack and cheddar work well)
    \item 10 \nameref{tortillas}, cut into sixths (rounded triangles)
\end{itemize}

\subsection*{Hardware}
\begin{itemize}
    \item Large mixing bowl
    \item Large Skillet
    \item Large stock pot
    \item 9x13 casserole dish
\end{itemize}
\clearpage

\subsection*{Instructions}
\begin{enumerate}
    \item Make a recipe of \nameref{creamOfMushroomSoup} in a large stock pot if you haven't yet, while starting chicken.
    \item Season 1,150 grams of chicken with 2 teaspoons salt and 1 teaspoon black pepper in a large mixing bowl.
    \item Fry seasoned chicken in vegetable oil in a large skillet. Do this in about 3 batches to allow room between pieces. Do this over medium-high heat to promote delicious fond.
    \item Fry the chicken for about 3 minutes per side, moving chicken to a clean bowl as batches finish.
    \item Place chicken in a 270F oven while finishing other batches and vegetables.
    \item Fry 1 small chopped onion in the chicken fat and fond.
    \item When onion has just gone translucent, add to Cream of Mushroom sauce.
    \item Fry 2 chopped bell peppers in 2 batches in the leftover fat and fond. Add a pinch of oil if running low.
    \item When a batch of peppers has fried for about 5 minutes, add it to the sauce.
    \item Add 120 grams chicken broth, 2 teaspoons cumin, 2 teaspoons chili powder, 1 Tablespoon kosher salt, and \( \frac{1}{2} \) teaspoons dried oregano, and 283 grams diced tomatoes to the sauce. Stir to combine.
    \item If the sauce is too thin at this point, you can thicken it by adding 1 teaspoon corn starch disolved into 1 Tablespoon warm water.
    \item If the sauce is ever too thin (rarely happens), you can add more chicken broth, water, etc.
    \item Cook sauce over medium heat, stirring occasionally. When the sauce is as thick as you like (fairly thick, ideally), it is ready.
    \item preheat oven to 350F
    \item Grease a 9x13 casserole dish.
    \item Place a small amount of the sauce on the bottom of the dish.
    \item Layer half the chicken in the dish.
    \item Layer half the tortillas, overlap is fine.
    \item Layer half the saucer over this.
    \item Layer half the cheese over this.
    \item Repeat the above 4 layers to finish the dish. 
    \item This can be covered and stored in the fridge at this point.
    \item Bake the casserole, uncovered at 350F for about 30 minutes until bubbling and hot (maybe 1 hour if cold from fridge).
\end{enumerate}

\subsection*{Notes}
\begin{itemize}
    \item King Ranch Casserole is a very common dish in South Texas.
    \item Traditionally this uses Ro*Tel tomatoes, relatively few spices, boiled chicken, and canned cream of mushroom and cream of chicken. While that is an easy way to make a dish, it lacks flavor. This method (while still realtively bland) is much tastier, and still retains the core elements of the Texan classic.
    \item My mother made this dish in the more traditional way growing up, and I loved it. This version is my love letter to that fond memory.
\end{itemize}
\end{multicols}
\clearpage