\section{Cigánypecsenye}
\label{ciganypecsenye}
Time: 1 \( \frac{1}{2} \) hours (10 minutes prep, 45 minutes making fries, 30+ minutes making bacon, 5 minute cooking pork chops)
Serves: 4

\begin{multicols}{2}
\subsection*{Ingredients}
\begin{itemize}
    \item 4 thin pork chops, bone in or out (about 1 \( \frac{1}{2} \) pounds)
    \item 6 cloves garlic, diced
    \item 1 \( \frac{1}{2} \) cups oil (vegetable works)
    \item 4 teaspoons Hungarian spicy paprika
    \item 2 teaspoon marjoram
    \item 4 teaspoon ground mustard seed
    \item 2 teaspoon coarse salt
    \item 1 teaspoon fresh cracked pepper
    \item 4 servings of French Fries (Recipe to be added someday)
    \item 4 slices of Szalonna (Hungarian style bacon), very thick cut, with rind
    \item Additional Hungarian spicy paprika to taste
\end{itemize}

\subsection*{Hardware}
\begin{itemize}
    \item Skillet
    \item Medium mixing bowl
\end{itemize}
\clearpage

\subsection*{Instructions}
\begin{enumerate}
    \item Put the 6 gloves of garlic, 1 \( \frac{1}{2} \) cups oil, 4 teaspoons of paprika, 2 teaspoons marjoram, 4 teaspoons ground mustard seed, 2 teaspoons salt, and 1 teaspoon of pepper into the mixing bowl.
    \item Mix thouroughly.
    \item Place the pork chops into the bowl, and make sure that there is oil and spices on all sides, and covering the pork chops.
    \item Allow the pork chops to soak in the spice-oil while preparing bacon, occasionally spooning spices from the bottom back on top.
    \item Take the 4 slices of thick cut szalonna, ensure that they have slits down the non-rind side, about every 1 inch.
    \item Place the slices in the skillet at low heat.
    \item Carefully raise the heat a small amount (between \( \frac{1}{2} \) and 1 notch) every three minute, while flipping the zsallona.
    \item Once the bacon is somewhat crispy, remove and set aside.
    \item Cook the pork chops (probably about 2 at a time) for about 2 minute at a side in the fat.
    \item Place all pork chops in a dish in the oven, covered, at low heat.
    \item Fry the french fries in the fat.
    \item Plate the fries, then place a pork chop on each pile.
    \item Sprinkle additional paprika to taste over the pork chop.
    \item Place a szallona slice on each pork chop.

\end{enumerate}

\subsection*{Notes}
\begin{itemize}
    \item Based partly on a trip to Budapest, partly on this recipe: \url{http://www.nosalty.hu/recept/egyszeru-ciganypecsenye}, with the following differences:
    \begin{itemize}
        \item Less oil is used.
        \item More paprika is added after cooking.
        \item Hard for me to say, as my Hungarian is basically nill.
    \end{itemize}
    \item FOR NEXT TIME: Try instead of ground mustard seed, cooking the chops without paprika, then spreading mustard and paprika on after cooking. This may be more authentic...
    \item "Cigánypecsenye" translates to "Gypsy Steak"
    \item "Szallona" is just Hungarian for "bacon", however when I use that word in this recipe I am referring to a particular style of bacon I cannot hardly find in the states. It it still made from the pork belly, but is basically entirely just fat and rind, with none of the bits of meat strips typical in American bacon. I recommend googling "ciganypecsenye bacon" for examples. Regular American bacon can be used instead, made similar to the \nameref{crispyBacon} recipe, but make it less crispy, so it cute with teh pork chop easier.
    \item The Szallona is cut along an edge to allow it to curl while cooking, and get more crispy edges, please see pictures online for examples.
    \item Be very careful cooking the mostly-fat bacon, as it can burn very quickly.
    \item If you manage to use the mostly-fat style szallona, I don't recommend eating the rind, instead eat each little slice off the rind.
    \item Concerning paprika: Hungarian papriak is (I'm quite convinced now) the best paprika in the world. While you can find it online, it's no substitute for the real thing. If you do not know where to get good paprika, please befriend a Hungarian (all of whom I've met are extraordinarily nice people) and ask them.
    \item I recommend "spicy" paprika, which is not terribly spicy really, it just means don't use "sweet" paprika, which is also not terribly sweet, they just taste different.

\end{itemize}
\end{multicols}
\clearpage