\section{Fried Chicken}
\label{friedChicken}
\setcounter{secnumdepth}{0}
Time: 11 hours (30 minutes prep, 9 hours brining, 1 hour 30 minutes cooking)
Serves: 4

\begin{multicols}{2}
\subsection*{Ingredients}
\begin{itemize}
    \item 1 whole fryer chicken, about 4-6 pounds, disjointed (8 pieces)
    \item 1 recipe of \nameref{chickenBrine}
    \item 3 cups buttermilk
    \item 3 \( \frac{1}{2} \) Tablespoons kosher salt
    \item 3 cups of all purpose flour
    \item 3 \( \frac{1}{2} \) Tablespoons paprika
    \item 2 Tablespoons cayenne pepper
    \item 1 Tablespoon black pepper, freshly ground
    \item 6-8 cups of vegetable oil, enough to make 1-2 inch deep pool in dutch oven.
\end{itemize}

\subsection*{Hardware}
\begin{itemize}
    \item 9x13 Baking dish with lid (or plastic wrap)
    \item wire rack
    \item 2 Mixing bowls
    \item 5-7 Quart dutch oven
    \item Frying thermometer
    \item Meat thermometer
\end{itemize}
\clearpage

\subsection*{Instructions}
\begin{enumerate}
    \item Disjoint chicken.
    \item Remove skin from breasts.
    \item Cut breasts into 3-4 large strips.
    \item Place all chicken pieces in 9x13 baking dish.
    \item Pour \nameref{chickenBrine} over the chicken.
    \item Cover with lid or plastic.
    \item Leave in refrigerator for about 4-5 hours.
    \item Turn all chicken over and leave in teh fridge for an additional 4 hours.
    \item Remove chicken to wire rack to drip dry a little, and ensure no peppercorns or bay leaves are stuck to the chicken pieces.
    \item Combine 3 cups buttermilk and 1 Tablespoon salt in one mixing bowl.
    \item Combine 2 \( \frac{1}{2} \) Tablespoons kosher salt, 3 cups of all purpose flour, 3 \( \frac{1}{2} \) Tablespoons paprika, 2 Tablespoons cayenne pepper, and 1 Tablespoon black pepper in another mixing bowl.
    \item Pour 6-8 cups oil in dutch oven to make a pool about 2 inches deep.
    \item Heat oil to 350F, about medium-high on most stoves.
    \item Drop chicken pieces in the flour, about 3-4 at a time. Use one hand to coat the chicken in flour.
    \item Drop the coated chicken in the buttermilk mixture.
    \item Use your other hand to coat the chicken in buttermilk.
    \item Drop the wet chicken back in the flour to coat a second time.
    \item Drop battered chicken in hot oil (oil should stay between 300F and 350F).
    \item Fry the chicken for about 5 minutes on the first side.
    \item Flip each piece of chicken. 
    \item Large dark meat should fry for an additional 15 minutes. White meat (wings and strips) should cook for about an additional 10 minutes.
    \item All meat should reach an internal temperature of 165F.
\end{enumerate}

\subsection*{Notes}
\begin{itemize}
    \item This is based on an Epicurious recipe, as seen here: \url{https://www.epicurious.com/recipes/food/views/brined-fried-chicken-352449}
    \begin{itemize}
        \item Main differences are amount of cayenne, measurements are given in a more sane manner, thermometer is used to ensure safe meat temperatures, and breasts are deboned and split into strips.
    \end{itemize}
    \item While this is a VERY different chicken than I had growing up, it's superior in most every way. Toto always made dry battered chicken, but brines and thick breading just makes birds better.
\end{itemize}
\end{multicols}
\clearpage