\section{Csirkepaprikás}
\label{csirkePaprikas}
\setcounter{secnumdepth}{0}
Time: 2 hours (30 minutes prep, 1 \( \frac{1}{2} \) hours cooking)
Serves: 4

\begin{multicols}{2}
\subsection*{Ingredients}
\begin{itemize}
    \item 2 medium sweet onions, minced
    \item 2 Tablespoons lard
    \item 1 3-4 pound whole chicken, disjointed, washed, and dried
    \item 1 large, ripe tomato, peeled and chopped
    \item 2 heaping tablespoons sweet paprika
    \item \( \frac{1}{2} \) cup water
    \item 1 teaspoon salt
    \item 1 green bell pepper, sliced thin
    \item 1 tablespoon flour
    \item 1 teaspoon cold water
    \item 2 Tablespoons sour cream
    \item 2 Tablespoons heavy cream
    \item 1 recipe of \nameref{galuska}
    \item Additional sour cream for topping
\end{itemize}

\subsection*{Hardware}
\begin{itemize}
    \item 5 Quart stock pot with a good lid
    \item Small mixing bowl
\end{itemize}
\clearpage

\subsection*{Instructions}
\begin{enumerate}
    \item Cook 2 diced onions in 2 Tablespoons lard until soft, about 5 minutes, over low heat in stock pot.
    \item Add disjointed chicken and 1 chopped tomato.
    \item Stir in 2 heaping Tablespoons sweet paprika, \( \frac{1}{2} \) cup water, and 1 teaspoon salt.
    \item Cook, covered, over low heat for 20 minutes.
    \item Remove the lid and allow to cook for an additional 10 minute uncovered. This will allow the water and tomato juice to reduce some.
    \item Remove chicken from the stock pot.
    \item Remove the stock pot from the heat.
    \item Mix 1 Tablespoon flour and 1 teaspoon cold water in small mixing bowl.
    \item Add 2 Tablespoons sour cream to the flour/water mixture, stir until smooth.
    \item Take about a half-ladle of liquid from the pot and stir into the sour cream mixture until smooth and liquidy. If it is not yet liquidy enough to pour, stir in another half-ladle.
    \item Pour sour cream mixture into stock pot.
    \item Return stock pot to low heat and add 1 sliced green bell pepper.
    \item Allow juice and vegetables to cook over low heat for 5 minutes, this will allow the peppers to soften a little.
    \item Replace chicken parts and cook until finished, covered, over low heat, about 20 minutes.
    \item Keep warm until ready to serve, then whip in 2 Tablespoons heavy cream.
    \item Serve over a helping of galuska, and top with additional sour cream.

\end{enumerate}

\subsection*{Notes}
\begin{itemize}
    \item The recipe is from Gearge Lang's "The Cuisine Of Hungary", 1971 edition. Page 278.
    \begin{itemize}
        \item Main differences are I have added water amounts to the recipe, doubled the paprika, added some steps dealing with mixing sauce with sour cream before mixing, added steps to make bell peppers softer, and clarified some steps.
    \end{itemize}
    \item This is often called "Paprika Chicken" or "Chicken Paprikash" in English. In Hungarian it's also called "Csirke Paprikás", or, when served on nokedli (another name for galuska) it can be called "Csirkepaprikás nokedlivel".
    \item I highly recommend using the best sweet paprika you can find. Ideally you've either been to Hungary recently and have the good stuff, or have friends from there that can occasionally get you some. Store-bought paprika works alright, but is less flavorful.
    \item Note that the use of heavy cream in addition to sour cream is less common in modern Hungarian cuisine, however this used to be more common, makes for a creamier sauce, and is quite delicious. If you leave out the heavy cream, I recommend adding a little more sour cream instead.
    \item I tend to make the galuska as I'm doing some of the cooking of the sauce, as that part is fairly inactive.
    \item You add some warm sauce to the sour cream to get it warmer before it gets hot. This should help prevent curdling, as curdled sour cream would be too strong and not mix in well with the sauce.
    \item This would also be delicious with a side of boiled potatoes.
\end{itemize}
\end{multicols}
\clearpage