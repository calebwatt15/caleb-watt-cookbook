\section{Rakott kelkáposzta}
\label{rakottKelkaposzta}
\setcounter{secnumdepth}{0}
Time: 2 hours (1 hour prep, 1 hour cooking)
Serves: 4

\begin{multicols}{2}
\subsection*{Ingredients}
\begin{itemize}
    \item 1 head of savoy cabbage, about 2 pounds
    \item 3 quarts water
    \item 1 teaspoon salt
    \item 1 medium sweet onions, diced
    \item salt to taste
    \item \( \frac{1}{4} \) cup vegetable oil
    \item 2 cloved garlic, minced
    \item 1 heaping tablespoon sweet paprika
    \item 1 pound ground pork
    \item 100ml water (just a little shy of \( \frac{1}{2} \) cup)
    \item 1 teaspoon vegetable oil
    \item \( \frac{3}{4} \) cup white rice
    \item salt to taste.
    \item 1 \( \frac{1}{2} \) cups water
    \item 1 \( \frac{1}{2} \) cups sour cream (higher fat content the better)
    \item Additional sour cream on the side
\end{itemize}

\subsection*{Hardware}
\begin{itemize}
    \item 6 quart stock pot
    \item Tongs
    \item Colander
    \item Skillet with lid (or foil to cover)
    \item Small pan
    \item 9x9 Baking dish
\end{itemize}
\clearpage

\subsection*{Instructions}
\begin{enumerate}
    \item Peel all the leaves off the head of cabbage that you can, trying to keep them whole if possible.
    \item Clean all of the leaves.
    \item Cut out the bulk of the stem from each leaf by cutting a thin "V" shap down it.
    \item Boil about \( \frac{1}{3} \) of the cabbage at a time in 3 quarts of water and 1 teaspoon salt.
    \item Allow the first batch of cabbage to boil until soft, about 15 minutes.
    \item Remove the cabbage from the boiling water with tongs, place in colander to drain.
    \item Continue cooking cabbage in this manner until it's all cooked, but also move on to other steps while doing that.
    \item Meanwhile, saute 1 diced onion with some salt to taste in \( \frac{1}{4} \) cup vegetable oil for about 9 minutes, until soft, over medium heat.
    \item Add 2 cloves minced garlic and cook for an additional minute.
    \item If the dish is looking really oily, drain some out, as the pork will make just a bit more, even being fairly lean.
    \item Add ground pork, 1 heaping Tablespoon sweet paprika, and salt and pepper to taste.
    \item Make sure the meat is cut into tiny pieces, not large chunks, as it will need to be spread later.
    \item Cook the meat mixture for about 5 minutes, stirring frequently.
    \item Add 100 ml water to the meat and stir to combine briefly.
    \item Cover the meat and allow to cook for about 15 minutes.
    \item Heat 1 teaspoon vegetable oil in a small pan over medium heat.
    \item Add \( \frac{3}{4} \) cup rice (salt to taste) and cook in the oil for about 2 minutes, stirring constantly.
    \item Add 1 \( \frac{1}{2} \) cups water to the rice, cover, and reduce heat to low.
    \item Preheat oven to 350F.
    \item Allow the rice to cook until all the water is absorbed, about 12-15 minutes.
    \item Turn heat off, but leave rice on warm stove once the water is absorbed.
    \item Grease the 9x9 baking pan.
    \item Make a layer of cabbage across the bottom, about 3 leaves thick.
    \item Spread half the rice over the cabbage.
    \item Spread half the delicious-smelling paprika meat over the rice.
    \item Make a layer of cabbage on top of that about 1-2 leaves thick.
    \item Spread the remaining rice over the cabbage.
    \item Spread the remaining meat over the rice.
    \item Make another layer of cabbage, about 2 leaves thick (or more if you have it and some room).
    \item Spread 1 \( \frac{1}{2} \) cups sour cream over the top (or whatever fits).
    \item Bake in the oven at 350F for about 30 minutes. Ideally some edges will have browned by then.
    \item Turn the oven to broil for 3-5 minutes, watching the sour cream carefully. You want bits of it to brown and become crispy, without burning the whole thing.
    \item Can be served with additional sour cream on the side.

\end{enumerate}

\subsection*{Notes}
\begin{itemize}
    \item The recipe is from Culinary Hungary, as seen here: \url{http://budapestcookingclass.com/hungarian-layered-savoy-cabbage-casserole-recipe-rakott-kelkaposzta/}
    \begin{itemize}
        \item Main differences are I have added water and oil amounts to the recipe, added more paprika, reduced the cabbage (as that's all that will fit in my baking pan), and added steps with broiling, as just baking did not brown my sour cream well (may be the type of sour cream that is available here).
    \end{itemize}
    \item This recipe was sent to me by my good friend, Csaba. He continues to help me cook proper Hungarian food (or near as can be made in the States, anyhow).
    \item The name "Rakott Kelkáposzta" means something like "Layered cabbage" near as I can tell, though the website calls it "Hungarian Layered Savoy Cabbage Cassarole".
    \item If you can find savoy, I highly recommend it. It tastes a little more earthy than regular green cabbage, and is prettier. Greeb cabbage works fine as well. I don't recommend any other kind of cabbage, as the red tastes too different, and the others have too much stem-to-leaf ratio.
    \item I highly recommend using the best sweet paprika you can find. Ideally you've either been to Hungary recently and have the good stuff, or have friends from there that can occasionally get you some. Store-bought paprika works alright, but is less flavorful.
\end{itemize}
\end{multicols}
\clearpage