\section{Étouffée}
\label{etouffee}
\setcounter{secnumdepth}{0}
Time: 1 hour 20 minutes (15 minutes prep, 1 hour 5 minutes cooking)
Serves: 4

\begin{multicols}{2}
\subsection*{Ingredients}
\begin{itemize}
    \item 1 recipe of \nameref{roux} (dark brown)
    \item 1 cup chopped onion
    \item 1 cup sliced celery
    \item 1 cup chopped green bell pepper
    \item 4 cloves garlic, minced
    \item 3 green onions, diced
    \item Salt, pepper, and red pepper to taste (for meat)
    \item 1 pound of meat (shrimp, chopped chicken, or crawfish tails(see notes for crawfish))
    \item Stock to cover meat (about 2 cups, this can be chicken, vegetable, or my preferred, seafood stock)
    \item \( \frac{1}{2} \) tablespoon apple cider vinegar
    \item 1 splash of hot sauce (such as Tobasco)
    \item \( \frac{1}{2} \) teaspoon salt
    \item \( \frac{1}{2} \) teaspoon black pepper
    \item Dash of red pepper
    \item 1 recipe of \nameref{whiteRice}
\end{itemize}

\subsection*{Hardware}
\begin{itemize}
    \item Dutch oven with lid
    \item Large spoon
\end{itemize}
\clearpage

\subsection*{Instructions}
\begin{enumerate}
    \item Make a dark brown \nameref{roux}.
    \item Throw in 1 cup onion, 1 cup celery, and 1 cup green bell pepper.
    \item Allow vegetables to cook until just barely soft (about 5 minutes). This will absorb most of the roux.
    \item Add in 4 cloves of garlic and 3 diced green onions.
    \item Season raw meat with salt, black pepper, and red pepper.
    \item Add raw meat to dutch oven.
    \item Cover food in stock, about 2 cups.
    \item Add \( \frac{1}{2} \) tablespoon apple cider vinegar, 1 splash of hot sauce, \( \frac{1}{2} \) teaspoon salt, \( \frac{1}{2} \) teaspoon black pepper, and a dash of red pepper.
    \item Stir to combine.
    \item Cover étouffée with lid and allow to cooker for about 20 minutes.
    \item Remove lid and allow meat to finish cooking, while the liquid reduces to the desired level.
    \item Taste and season food to desired level.
    \item Server over white rice.
\end{enumerate}

\subsection*{Notes}
\begin{itemize}
    \item This is my own recipe.
    \item "Étouffée" means "smothered", which is typically how the meat is cooked in the dish.
    \item For crawfish, assuming you are starting with cooked tails (such as leftovers from a boil) then add them in toward the last 5 minutes of cook-time, to prevent overcooking.
    \item Shrimp can be added in the last 10 minutes of cooking, as it will cook much quicker than chicken breast, and gets tough if overcooked.
    \item I prepare a nice roux base and use stock, which is not always seen in cajun recipes, however I refuse to use tomatoes.
    \item You can use a red wine for acidity rather than apple cider vinegar, however I've avoided this so far as I'm not much good at judging wine.
    \item You can use a lighter roux if you like, however I prefer a really dark roux.
\end{itemize}
\end{multicols}
\clearpage