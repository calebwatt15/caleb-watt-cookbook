\section{Pierogi}
\label{pierogi}
\setcounter{secnumdepth}{0}
Time: 1 hour (10 minutes prep, 30 minutes resting, 20 minute cooking)
Serves: 4 (about 16 pierogi)

\begin{multicols}{2}
\subsection*{Ingredients}
\begin{itemize}
    \item 4 ounces (about \( \frac{1}{2} \) cup) water
    \item 4 ounces butter
    \item 11 ounces all purpose flour (10 ounces for optional sweet dough)
    \item pinch of kosher salt
    \item 1 egg
    \item Choice of fillings from \nameref{pierogiFillings}
    \item 2 quarts water for boiling
    \item Sour cream (for topping)
    \item Optionally, for sweet dough, reduce flour as stated above, then add 1 Tablespoon confectioner's sugar. Keep 2 Tablespoons confectioner's sugar for topping.
    \item Optionally, for sweet pierogi, serve with a lightly whipped cream, or \nameref{whippedCream}
\end{itemize}

\subsection*{Hardware}
\begin{itemize}
    \item Small sauce pan
    \item Mixing bowl
    \item Plastic wrap or damp towel
    \item Rolling pin
    \item Thin lipped glass, about 3-4 inches across
    \item Large stock pot
    \item Skillet
    \item Slotted spoon
\end{itemize}
\clearpage

\subsection*{Instructions}
\begin{enumerate}
    \item Bring 4 ounces (about \( \frac{1}{2} \) cup) water to a boil in a small sauce pan.
    \item Place 1 ounce butter in the water to melt.
    \item Combine 11 ounces flour and a pinch of kosher salt in a mixing bowl. (10 ounces for a sweet dough, then add 1 Tablespoon confectioner's sugar).
    \item Slowly drizzle in water and and butter into dry mix while stirring and combining with a spoon.
    \item Place egg into dough mixture.
    \item Knead by hands until smooth, about 5 minutes.
    \item Allow dough to rest for 30 minutes before using.
    \item Bring about 2 quarts of water to a boil in a stock pot.
    \item Roll dough out until it's very thin, about a tenth of an inch thick (3 mm).
    \item Use a thin lipped glass to cut out circles from the dough, you should get 16 or so dough circles.
    \item Stretch out a circle a little, and add a heaping teaspoon of filling into the middle.
    \item Fold over into a half circle and crimp the edges.
    \item Repeat until all dumplings are put together.
    \item Place dumplings into stock pot of boiling water.
    \item Allow dumplings to cook until they float to the top, then cook for an additional 2 minutes.
    \item While dumplings are cooking, bring 3 ounces butter to medium high heat in a skillet.
    \item As the dumplings finish, drain briefly, then place in hot butter to fry, about 1-2 minutes per side. (No need to fry optional sweet pierogi, just drain, add confectioner's sugar, then serve with cream.)
    \item Serve with sour cream (for savory pierogi).
\end{enumerate}

\subsection*{Notes}
\begin{itemize}
    \item This is based on the pierogi recipe as seen here: \url{https://www.kwestiasmaku.com/przepis/ciasto-na-pierogi}.
    \begin{itemize}
        \item Main difference is I converted grams to ounces, and increased the recipe size slightly.
    \end{itemize}
    \item The best pierogi I've had were at Pierogi Plus, outside of Pittsburgh, PA. If you are ever even close, please do yourself a favor and order a dozen (be warned, they are rather large).
    \item If you make \nameref{polishDillBrusselsprouts} or \nameref{mizeria} as a side, you can dip pierogi in leftover dill sauce instead of sour cream. It's delicious.
    \item You will end up with a bunch of extra dough with the holes in it. You can try and rest it, re-roll, and then make about 2 more really large pierogi. I've done this, but the dough is really thick and tough. Not too bad, but not great. Worth not wasting dough, I reckon.
\end{itemize}
\end{multicols}
\clearpage