\section{Slow Cooker Pork Carnitas}
\label{slowCookerPorkCarnitas}
Time: 6 hours (15 minutes prep, 5 hours inactive cooking, 45 minutes cooking)
Serves: 6

\begin{multicols}{2}
\subsection*{Ingredients}
\begin{itemize}
    \item 1 medium onion, shopped into chunks
    \item 2 cinnamon sticks
    \item 1 tablespoon oregano
    \item \( \frac{1}{2} \) tablespoon cumin
    \item 2 teaspoons salt
    \item 1 teaspoon pepper
    \item 3-4 pound boneless pork butt, cut into 3 inch chunks
    \item 6 cloves of garlic, cut in halves
    \item 2 jalapeños, stemmed, quartered, and deseeded
    \item 1 large orange, cut in half (not peeled)
    \item 1 cup of lard, split into thirds
\end{itemize}

\subsection*{Hardware}
\begin{itemize}
    \item 6 quart slow cooker or similarly sized dutch oven
    \item Small bowl for mixing spices
    \item 2 forks (for shredding meat)
    \item Skillet
\end{itemize}
\clearpage

\subsection*{Instructions}
\begin{enumerate}
    \item Put the onion chunks and cinnamon sticks into the bottom of the slow cooker.
    \item In a small bowl, combine 1 tablespoon oregano, \( \frac{1}{2} \) tablespoon cumin, 1 teaspoon salt, and 1 teaspoon pepper.
    \item Shake all pork chunks in spice mixture.
    \item Place the meat and the remaining spices into the slow cooker, on top of the rough onion chunks.
    \item Place the 6 garlic cloves and jalapeno chunks on top of the meat chunks.
    \item Squeeze the oranges to get most of the juice into the slow cooker.
    \item Place the orange halves on top of the whole kit and caboodle.
    \item Cook covered on low for 6-7 hours, until tender (to the point where it could be shredded with forks.)
    \item Remove pork from juice and vegetables.
    \item Strain the vegetables from the juice, saving the juice for frying.
    \item Shred the meat with two forks.
    \item Place \( \frac{1}{3} \) cup of lard into a skillet and heat to medium-high.
    \item When the lard is hot, place about \( \frac{1}{3} \) of the shredded meat spaced out in the skillet, along with \( \frac{1}{4} \) cup of the juice from the slow cooker.
    \item Allow the meat to become crispy on one side.
    \item Serve the first \( \frac{1}{3} \) of the meat while cooking the next batch. Continue this way until all meat is cooked.

\end{enumerate}

\subsection*{Notes}
\begin{itemize}
    \item Traditionally served with tortillas, cilantro, and raw onion pieces as a small taco.
    \item Works well in any taco, nacho, quesadilla, or burrito.
    \item These are not a traditional carnitas recipe, as the pork is not braised in lard initially, however the flavors, spices, and final result are very close and very good.
    \item For a more traditional recipe, see Traditional Pork Carnitas. This recipe is designed to be very simple to cook.
    \item If you do not wish to serve the entirety of the carnitas at once, bag portions of it with the juice from the slow cooking process. Freeze bags of this, then to cook, allow to thaw, then fry the meat and juice that was frozen with some lard.
    \item You can use vegetable oil rather than lard, if you must.
    \item I tend to by a 7-8 pound roast, cut it in half, and make both this and a \nameref{slowCookerBostonButt}.

\end{itemize}
\end{multicols}
\clearpage