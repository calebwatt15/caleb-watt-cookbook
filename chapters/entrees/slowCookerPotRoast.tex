\section{Slow Cooker Pot Roast}
\label{slowCookerPotRoast}
\setcounter{secnumdepth}{0}
Time: 9 hours (30 minutes prep, 8 hours inactive cooking, 30 minutes active)
Serves: 4-6

\begin{multicols}{2}
\subsection*{Ingredients}
\begin{itemize}
    \item 2 pounds carrot sticks (or carrots, cut into small pieces)
    \item 4 pound chuck roast (the fattier the better)
    \item Salt, pepper, and garlic powder to taste
    \item About 1 cup of flour (for covering the roast)
    \item \( \frac{1}{2} \) cup oil for pan frying (bacon grease, vegetable oil... it's all good)
    \item \( \frac{1}{4} \) cup chopped dill
    \item \( \frac{1}{8} \) cup chopped oregano
    \item \( \frac{1}{2} \) cup butter (whole stick)
    \item 6-8 peporoncinis plus some juice (maybe \( \frac{1}{4} \) cup), destemmed
\end{itemize}

\subsection*{Hardware}
\begin{itemize}
    \item Skillet
    \item Slow cooker
    \item Cutting board
    \item Two forks (for shredding)
\end{itemize}
\clearpage

\subsection*{Instructions}
\begin{enumerate}
    \item Place 2 pounds of carrot sticks in slow cooker (this should include just a bit of water, if not, add some water, maybe \( \frac{1}{4} \) cup)
    \item Salt, pepper, and garlic powder carrot sticks to taste.
    \item Take your roast, add salt, pepper, garlic powder, and flour to every side.
    \item Heat up \( \frac{1}{2} \) cup oil in a skillet
    \item Sear all sides of the roast in the hot oil for 1-2 minutes each, just to brown the meat.
    \item Place seared roast on top of carrots.
    \item Place \( \frac{1}{2} \) cup butter (1 stick) on top of roast
    \item Add salt, pepper, garlic powder, \( \frac{1}{4} \) cup dill, \( \frac{1}{8} \) cup oregano on top of butter.
    \item Chop stems off of pepperoncini.
    \item Place pepperoncini around the butter stick, on top of the roast.
    \item Add a splash of the pepperoncini juice onto the roast
    \item Close slow cooker and cook on low for 7-8 hours. Flip at the halfway-way point (the butter will be melted, don’t worry about the pepperoncini falling in the mix, it’s all good.)
    \item Remove meat from juice when tender, shred with forks.
    \item Remove carrots and pepperoncinis from juice.
    \item Place all juice into a skillet at medium-high heat, cook down a little to remove some of the water from the roast.
    \item Add \( \frac{1}{4} \) to \( \frac{1}{2} \) cup flour and stir into the grease to create a light beef gravy.
    \item Chop pepperoncinis into bits.
    \item Recombine carrots, pepperoncinis, shredded roast, and gravy.
\end{enumerate}

\subsection*{Notes}
\begin{itemize}
    \item Some people do not like pepperoncinis. If so, use a splash of some other acid, such as distilled white vinegar.
    \item Searing the roast is not strictly necessary, however I do recommend it. Some people tell you to sear a roast to “lock in moisture”, that is nonsense. I sear it for 2 reasons. 1, I like the taste of browned meat, and it’s easier to brown it in a skillet than cooking in it’s own juice. 2, I like the flour on the outside to just start cooking with the hot oil, it’s helps turn all the future-juice into a more gravy-like substance.
    \item While the addition of fat and pepperoncinis is similar to a “Mississippi Roast”, I refuse to use packaged dressing mixes. I do like the additional fat, acid, and herb flavors though.
\end{itemize}
\end{multicols}
\clearpage