\section{Cornish Pasty}
\label{cornishPasty}
\setcounter{secnumdepth}{0}
Time: 1 hour 20 minutes (30 minutes prep, 50 minutes baking)
Serves: 4

\begin{multicols}{2}
\subsection*{Ingredients}
\begin{itemize}
    \item 3 pie crust recipes (see \nameref{pieCrust})
    \item 16 ounces potatoes, diced
    \item 8 ounces rutabaga, diced
    \item 4 ounces onion, diced
    \item 16 ounces skirt steak, sliced into \( \frac{1}{2} \) inch wide strips, 3 inches long.
    \item \( \frac{1}{2} \) teaspoon salt
    \item \( \frac{1}{2} \) teaspoon black pepper
    \item 1 egg
    \item 3 teaspoons water
    \item 4 Tablespoons all purpose flour
    \item 2 ounces butter or clotted cream
\end{itemize}

\subsection*{Hardware}
\begin{itemize}
    \item Mixing bowl
    \item Small bowl
    \item Pastry brush (or your fingers)
    \item Fork for crimping
    \item Baking sheet
\end{itemize}
\clearpage

\subsection*{Instructions}
\begin{enumerate}
    \item Ensure that the triple pie crust recipe is ready, but split it into four equal parts and allow to rest in the fridge.
    \item Pre-heat oven to 425F.
    \item combine 16 ounces diced potatoes, 8 ounces diced rutabaga, 4 ounces diced onion, 16 ounces skirt steak strips, \( \frac{1}{2} \) teaspoon salt, and \( \frac{1}{2} \) teaspoon black pepper in a mixing bowl.
    \item Combine 1 egg and 1 teaspoon water in a small bowl.
    \item Roll out each piece of the pastry crust to a thin round, approximately 7 inches across.
    \item Spread about \( \frac{1}{4} \) of the filling, or as much as will fit on just less than hald the crust on each piece of pastry crust.
    \item Sprinkle 1 Tablespoon of flour over each pasty filling.
    \item Cut 2 ounces of butter into many small pieces, spread several pieces into each pasty until it is all used.
    \item Brush the entire circumference of the crust with the egg wash.
    \item Fold over the pastry crust, and crimp along the edge.
    \item Poke a small hole (less than \( \frac{1}{2} \) inch across) in the top of the crust.
    \item Use the rest of the egg wash to wash each pasty.
    \item Place pasties on a baking sheet and bake at 425F.
    \item About 10 minutes into the baking time, use a small spoon to drop in about \( \frac{1}{4} \) - \( \frac{1}{2} \) teaspoon of water into each pasty.
    \item Bake for about 45-50 minutes total, until the pasties are golden brown on top, and the meat is cooked through.
    \item Serve hot or cold.
\end{enumerate}

\subsection*{Notes}
\begin{itemize}
    \item This is my own Cornish Pasty recipe, though it's pretty standard.
    \item Pasties are common in parts of Michigan, however they have deviated slightly from this. Often using carrots in addition to or instead of rutabagas.
    \item My main Michigan friend (Ben Loula) insists that up in those parts, a pasty is consumed with either a good brown gravy, or ketchup. Any given person eats the pasty with either one or the other, and anyone who eats with the other condiment is insane.
    \item Ketchup for life.
    \item Technically in order for a pasty to be "Cornish" it must be made in the Cornish area, with ingredients from local cows, and the meat must be at least 12.5\% of the total weight. Given that, all of mine are technically just pasties, not Cornish Pasties, but my style and ingredients are fairly authentic (though my pastry is better, I'd wager.)
\end{itemize}
\end{multicols}
\clearpage