\section{Tordai Kelpástétom}
\label{tordaiKelpastetom}
\setcounter{secnumdepth}{0}
Time: 3 hours (1 hour prep, 2 hours cooking)
Serves: 6

\begin{multicols}{2}
\subsection*{Ingredients}
\begin{itemize}
    \item 2 pounds of savoy cabbage (about 1 medium head)
    \item 3 quarts water
    \item 1 teaspoon salt
    \item 1 dinner roll (see )
    \item \( \frac{1}{2} \) cup milk
    \item 1 cup water
    \item Pinch of salt
    \item \( \frac{1}{2} \) cup uncooked white rice
    \item 1 pound ground pork
    \item 1 medium sweet onion, minced
    \item 2 Tablespoons lard
    \item 1 clove garlic, minced
    \item 2 eggs
    \item \( \frac{1}{2} \) teaspoon dried marjoram
    \item \( \frac{1}{4} \) teaspoon black pepper
    \item 1 teaspoon paprika (sweet or spicy)
    \item \( \frac{1}{2} \) teaspoon salt
    \item 4 Tablespoons bacon drippings
    \item \( \frac{1}{4} \) pound double-smoked bacon (about 3 thick-cut strips)
    \item 8-10 Tablespoons chicken broth
    \item 1 Tablespoon paprika (sweet or spicy)
    \item \( \frac{1}{2} \) cup chicken broth
    \item 1 Tablespoon flour
    \item \( \frac{1}{2} \) cup sour cream
\end{itemize}

\subsection*{Hardware}
\begin{itemize}
    \item Stock pot
    \item Small pot with lid
    \item skillet
    \item 5+ Quart dutch oven
\end{itemize}
\clearpage

\subsection*{Instructions}
\begin{enumerate}
    \item Cut out core of cabbage and wash.
    \item Cook the cabbage in enough lightly boiling water to float in (about 3 quarts) and about 1 teaspoon salt until about half soft, maybe 10 minutes.
    \item Seperate the cabbage into leaves.
    \item Cut out thick stems from cabbage leaves.
    \item Leave the dinner roll in \( \frac{1}{2} \) cup milk, rotating it periodically to soak milk into all parts.
    \item Bring 1 cup water to boil with a pinch of salt in a small pot.
    \item Add \( \frac{1}{2} \) cup rice to the boiling water in the small pot.
    \item Reduce heat to low and cover with a lid.
    \item Cook the rice until the water is absorbed, about 12 minutes.
    \item Turn off the heat, but leave the rice on the warm element.
    \item Shred the soaked dinner roll.
    \item Mix cooked rice, 1 pound ground pork and shredded roll.
    \item Saute 1 minced onion in 2 Tablespoons lard in a skillet for about 4 minutes over medium heat.
    \item Add 1 clove minced garlic to the onions and saute for 1 more minute.
    \item Mix onions and garlic into the pork mixture.
    \item Add 2 eggs to the pork mixture.
    \item Add \( \frac{1}{2} \) teaspoon dried marjoram, \( \frac{1}{4} \) teaspoon black pepper, 1 teaspoon paprika, and \( \frac{1}{2} \) teaspoon salt to the pork mixture.
    \item Mix to fully combine all parts of the pork mixture.
    \item Coat the inside of the dutch oven with about 4 Tablespoons of bacon drippings.
    \item Make a layer on the bottom of the dutch oven with about \( \frac{1}{3} \) of the cabbage leaves.
    \item Spread half of the pork mixture, completely covering the cabbage. Press the pork mixture down well to make a solid layer.
    \item Make another layer with another \( \frac{1}{3} \) of the cabbage leaves.
    \item Spread the remaining pork mixture over this layer of cabbage, again pressing it down all over to make a very compact meatloaf-like structure.
    \item Spread the remaining cabbage over the top in a flat layer. Press it all down as well as you can.
    \item Place the bacon strips over the top of the whole cabbage casserole.
    \item Bake in a 350F oven, and baste with a couple tablespoons of chicken broth about 2-3 times while baking.
    \item Leave the casserole in the oven and switch it to high broil for about 3 minutes to help crisp the bacon.
    \item Turn the "pâté" from the dutch oven.
    \item If you like, you can make a sauce as follows (though I find the dish is often moist enough this is not strictly necessary.)
    \item Put the dutch oven on the stove top at medium heat with the leftover juice and drippings from the "pâté".
    \item Scrape up the drippings in the juice, add 1 Tablespoon flour and stir for a few minutes over medium heat to form a sort of blond roux.
    \item Add 1 Tablespoon paprika and \( \frac{1}{2} \) cup chicken broth.
    \item Bring to a boil while stirring, and allow to boil for about 4 minutes.
    \item Remove from heat.
    \item Add a few tablespoons of the warm sauce to \( \frac{1}{2} \) cup sour cream to bring the sour cream up to heat.
    \item If the sour cream is a liquid that can be easily poured, add it to the sauce. Otherwise continue to add a few tablespoons of sauce and mix it in each time.
    \item Combine the sour cream mixture with the sauce.
    \item Serve cross-wise slices of the "pâté" (so there are three thin pieces of bacon on each slice) with some sauce poured over it.

\end{enumerate}

\subsection*{Notes}
\begin{itemize}
    \item The recipe is from Gearge Lang's "The Cuisine Of Hungary", 1971 edition. Page 311.
    \begin{itemize}
        \item Main differences are I have added water amounts to the recipe, clarified instructions, I use raw pork versus cooked pork, and changed sauce base from water to chicken broth.
    \end{itemize}
    \item The name "Tordai Kelpástétom" means something like "Todai-style Cabbage pâté" though it is not technically a pâté, this is what the Hungarian apparently call it in the Southeast region from where it originates.
    \item While called a pâté, I consider more of a meatloaf made with pork, and layered between cabbage leaves, as the process of adding bits of bread to meat and spices is very similar to American meatloaf.
    \item If you can find savoy, I highly recommend it. It tastes a little more earthy than regular green cabbage, and is prettier. Green cabbage works fine as well. I don't recommend any other kind of cabbage, as the red tastes too different, and the others have too much stem-to-leaf ratio.
    \item I highly recommend using the best paprika you can find. Ideally you've either been to Hungary recently and have the good stuff, or have friends from there that can occasionally get you some. Store-bought paprika works alright, but is less flavorful.
\end{itemize}
\end{multicols}
\clearpage