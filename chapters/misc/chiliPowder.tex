\section{Chili Powder}
\label{chiliPowder}
\setcounter{secnumdepth}{0}
Time: 30 minutes (5 minutes prep, 3 minutes cooking, 20 minutes cooling, 2 minutes blendingn)
Serves: About 2 ounces of chili powder (2-4 pots of chili)

\begin{multicols}{2}
\subsection*{Ingredients}
\begin{itemize}
    \item 3 ancho chiles, stemmed, seeded, and chopped
    \item 3 cascabel chiles, stemmed, seeded, and chopped
    \item 3 dried arbol chiles, stemmed, seeded, and chopped
    \item 1 Tablespoon ground cumin
    \item 2 Tablespoons garlic powder
    \item 1 Tablespoon dried oregano
    \item 1 teaspoon paprika
\end{itemize}

\subsection*{Hardware}
\begin{itemize}
    \item Skillet
    \item Blender
\end{itemize}
\clearpage

\subsection*{Instructions}
\begin{enumerate}
    \item Place all 9 chiles in skillet with 1 Tablespoon ground cumin.
    \item Cooker over medium-high heat for about 3-4 minutes just until the cumin gets toasty. Especially with ground cumin (as opposed to whole seeds) this must keep moving to keep from burning.
    \item Set aside chiles and cumin to cool completely, about 20 minutes.
    \item Place chiles and cumin into blender, along with 2 Tablespoons garlic powder, 1 Tablespoon dried oregano, and 1 teaspoon paprika.
    \item Blend all ingredients until it is a fine powder consistency.
    \item Allow powder to settle in blender for at least a minute before opening.
\end{enumerate}

\subsection*{Notes}
\begin{itemize}
    \item This is based on Alton Brown's recipe, as seen here: \url{https://www.foodnetwork.com/recipes/alton-brown/abs-chili-powder-recipe-1943055}
    \begin{itemize}
        \item Main differences are the use of ground cumin (as I didn't have seeds on hand), and Hungarian paprika rather than smoked paprika (again, I used what I had on hand).
    \end{itemize}
    \item This is more expensive than buying chili powder, but it's WAY more flavorful and delicious. Highly recommend trying it.
    \item This should keep for about 6 months in an air-tight container.
\end{itemize}
\end{multicols}
\clearpage