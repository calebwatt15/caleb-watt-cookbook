\section{Chicken Brine}
\label{chickenBrine}
\setcounter{secnumdepth}{0}
Time: 5 minutes (5 minutes prep)
Serves: About 6 cups, enough for 1 disjointed chicken.

\begin{multicols}{2}
\subsection*{Ingredients}
\begin{itemize}
    \item 5 cups water
    \item \( \frac{1}{2} \) cup granulated sugar
    \item \( \frac{1}{4} \) cup salt
    \item 2 bay leaves, torn into large pieces
    \item \( \frac{1}{2} \) Tablespoon black pepper corns
\end{itemize}

\subsection*{Hardware}
\begin{itemize}
    \item Mixing bowl
\end{itemize}
\clearpage

\subsection*{Instructions}
\begin{enumerate}
    \item Place all ingredients in a mixing bowl.
    \item Mix to combine.
    \item Use on a disjointed chicken or refirgerate immediately.
\end{enumerate}

\subsection*{Notes}
\begin{itemize}
    \item This is based on an Epicurious recipe, as seen here: \url{https://www.epicurious.com/recipes/food/views/brined-fried-chicken-352449}
    \begin{itemize}
        \item Main differences are measurements for the water, more bay leaves, more pepper corns, no coriander.
    \end{itemize}
    \item I highly recommend brining a chicken before making \nameref{friedChicken}. It keeps the bird super moist and flavorful.
    \item A brine works by moving the salty solution from areas of high salt concentration (a pan) into aread of low salt concentration (the bird). This allows the bird to retain moisture, as well as get additional flavor before cooking.
    \item I brine a whole disjointed chicken for about 8-9 hours before frying.
\end{itemize}
\end{multicols}
\clearpage