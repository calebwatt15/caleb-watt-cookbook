\section{Tex Mex Holy Trinity}
\label{texMexHolyTrinity}
\setcounter{secnumdepth}{0}
Time: 10 minutes (5 minutes prep, 5 minutes crushing)
Serves: 1 recipe of tex mex holy trinity

\begin{multicols}{2}
\subsection*{Ingredients}
\begin{itemize}
	\item 3 cloves garlic, peeled
	\item 1 Tablespoon water
    \item 1 \( \frac{1}{2} \) teaspoons whole cumin seeds
    \item 1 \( \frac{1}{2} \) teaspoons whole black peppercorns
\end{itemize}

\subsection*{Hardware}
\begin{itemize}
    \item Molcajate
\end{itemize}
\clearpage

\subsection*{Instructions}
\begin{enumerate}
    \item Combine all ingredients in the molcajete.
    \item Crush until it is all a nice smooth paste.
    \item Store for up to a month in the fridge.
\end{enumerate}

\subsection*{Notes}
\begin{itemize}
    \item This is Sylvia Casares' recipe, as seen in The Enchilada Queen Cookbook (and more than 20 years living on the border).
    \begin{itemize}
        \item That being said, this is a super common recipe, and the basis for a huge part of Tex Mex flavors. This recipe is very standard.
    \end{itemize}
    \item Rosie Flores (a friend) told me that you are supposed to ground the dry parts first, then add the garlic and water. I've tried both ways and can not taste a difference, so I just do it all at once to save some time.
\end{itemize}
\end{multicols}
\clearpage