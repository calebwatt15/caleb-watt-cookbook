\section{Marinara Sauce}
\label{marinaraSauce}
\setcounter{secnumdepth}{0}
Time: 1 hour 30 minutes (30 minute prep, 1 hour cooking)
Serves: 4

\begin{multicols}{2}
\subsection*{Ingredients}
\begin{itemize}
    \item 4 Tablespoons olive oil
    \item 56 ounces San Marzano (or other low-acid sauce tomatos), canned or freshly-peeled
    \item \( \frac{1}{2} \) medium yellow onion, diced
    \item 8 cloves garlic, minced
    \item 2 bay leaves
    \item 2 sprigs basil
    \item \( \frac{1}{3} \) cup chopped basil leaves
    \item Salt to taste
    \item Pinch of baking soda (depending on the tomatoes in use)
\end{itemize}

\subsection*{Hardware}
\begin{itemize}
    \item Stock pot
\end{itemize}
\clearpage

\subsection*{Instructions}
\begin{enumerate}
    \item If you are using fresh tomatoes, you will need to boil them for 1 minute, then dunk them in iced water, then peel the skins off.
    \item Heat 4 Tablespoons olive oil in a stock pot over medium-low heat.
    \item Add \( \frac{1}{2} \) diced yellow onion.
    \item Allow the onions to cook, stirring occasionally, until soft, about 10 minutes.
    \item Add 8 cloves minced garlic and allow to cook for about 30 seconds.
    \item Crush the tomatoes into the pot with your hand.
    \item Add 2 bay leaves and 2 sprigs of basil.
    \item Bring the sauce to a simmer of medium-low heat.
    \item Cover loosely with a lid and allow to cook until thickened, stirring occasionally, for about an hour.
    \item Salt to taste when the sauce is at your desired thickness.
    \item If the sauce is too acidic, it can be neutralized with a small pinch of baking soda.
    \item Remove the bay leaves and basil sprigs.
    \item Add a drizzle of olive oil and the \( \frac{1}{3} \) cup chopped basil on top of the sauce.
    \item Serve over pasta, such as \nameref{eggNoodles}.

\end{enumerate}

\subsection*{Notes}
\begin{itemize}
    \item This is based on Ciao Florentina's recipe, as seen here: \url{https://ciaoflorentina.com/best-marinara-sauce-recipe/}
    \begin{itemize}
        \item Main difference is notes on fresh tomatoes, double bay leaves, and half basil sprigs, and the addition of baking soda if required.
    \end{itemize}
\end{itemize}
\end{multicols}
\clearpage