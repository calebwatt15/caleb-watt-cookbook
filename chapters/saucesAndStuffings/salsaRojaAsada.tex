\section{Salsa Roja Asada}
\label{salsaRojaAsada}
\setcounter{secnumdepth}{0}
Time: 25 minutes
Serves: 3 cups

\begin{multicols}{2}
\subsection*{Ingredients}
\begin{itemize}
    \item 2 Tablespoon vegetable oil
    \item 20 ounces (about 4) whole, ripe, red tomatoes
    \item 4 (about 2 \( \frac{1}{2} \) ounces) whole chiles serranos
    \item 1 clove garlic (about 1 ounce), still in it's skin
    \item 5 ounces (about \( \frac{1}{2} \)) onion
    \item \( \frac{1}{4} \) ounce parsley
    \item 1 teaspoon kosher salt
\end{itemize}

\subsection*{Hardware}
\begin{itemize}
    \item Cast iron skillet (or comal, if you have one)
    \item Knife and cutting board
    \item Blender (or molcajete, if you have one)
\end{itemize}
\clearpage

\subsection*{Instructions}
\begin{enumerate}
    \item Heat cast iron skillet over medium-low heat.
    \item Add 2 Tablespoon vegetable oil, 20 ounces whole tomatoes, 4 serranos, and 1 garlic clove (in it's skin) to the skillet.
    \item Cook over medium-low heat while constantly moving for 5 minutes.
    \item Add 5 ounces onion (about a half an onion) in 1 or 2 large chunks.
    \item Keep everything moving for another minute or two, then allow peppers to sit on one side at a time for about a minute each.
    \item The peppers should form heat spots. After about another 3 or so minute, remove everything from the skillet.
    \item Peel tomatoes.
    \item Allow everything to cool for about 10 minutes.
    \item Peel garlic.
    \item Roughly chop all vegetables (into quarters), and remove stems from the serranos.
    \item Place all vegetables into the bowl of a blender or molcajete.
    \item Blend on low or medium until it is all incorporated.
    \item Add \( \frac{1}{4} \) ounce parsley and 1 teaspoon kosher salt, blend to combine.
    \item Chill then serve.
\end{enumerate}

\subsection*{Notes}
\begin{itemize}
    \item This is my own salsa roja recipe, but it's fairly standard. Nothing too fancy going on.
    \item Salsas can come in three varieties: asada (roasted), cocida (cooked, usually boiled), and cruda (raw). I really like all three.
    \item 10 minutes is enough cook time to get just a little char on the chiles, and cook the tomatoes a little, but not all the way through.
    \item Any salsa roja should use "jitomate" (red tomatoes) as the base, with added garlic, chiles, onion, and salt. Sometimes pepper. Feel free to add pepper if you like.
    \item While I grew up in South Texas where the bulk of the salsa was boiled then chilled, I really also like roasted salsa. Even still, I chill all my salsa. Hot salsa is just inferior (I'm looking at you, Houston.)
\end{itemize}
\end{multicols}
\clearpage