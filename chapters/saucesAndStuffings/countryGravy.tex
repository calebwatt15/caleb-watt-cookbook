\section{Country Gravy}
\label{countryGravy}
\setcounter{secnumdepth}{0}
Time: 20 minutes cooking
Serves: 4

\begin{multicols}{2}
\subsection*{Ingredients}
\begin{itemize}
    \item Some sort of meat, usually ground sausage.
    \item 3 tablespoons meat drippings (ou'll get this from the meat)
    \item \( \frac{1}{4} \) cup all purpose flour
    \item 2 cups milk
    \item salt and pepper to taste
\end{itemize}

\subsection*{Hardware}
\begin{itemize}
    \item Skillet
    \item Spoon or whisk
\end{itemize}
\clearpage

\subsection*{Instructions}
\begin{enumerate}
    \item Cook a meat of some sort in a skillet.
    \item Set aside the meat.
    \item Throw out all but 3 tablespoons of the drippings.
    \item Add in the \( \frac{1}{4} \) cup of flour, and cook it like a roux. You want the flour cooked into the fat just a little, but not as dark as a cajun roux.
    \item Once the flour and fats are cooked together, slowly add in the milk, about \( \frac{1}{4} \) cup at a time. Make sure the milk is fully incorporated, hot, and begins to thicken before adding in another \( \frac{1}{4} \) cup. This usually takes 3-5 minutes each.
    \item Once all of the milk is incorporated and has thickened a little, remove from heat.
    \item Continue to stir while gravy cools. It will thicken a little more while cooling down.
\end{enumerate}

\subsection*{Notes}
\begin{itemize}
    \item Amounts of flour and fats are rough counts, based on cooking with my mom growing up. She never really measured, just eye balled it. You get a feel for the right amounts after enough gravies.
    \item You can easily stir sausage back into the final gravy to get a sausage gravy.
    \item This can be made with most any meat drippings. I made it with brisket drippings once and it tasted like liquid brisket. Highly recommend trying it.
\end{itemize}
\end{multicols}
\clearpage