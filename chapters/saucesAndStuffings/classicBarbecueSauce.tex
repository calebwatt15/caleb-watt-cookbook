\section{Classic Barbecue Sauce}
\label{classicBarbecueSauce}
\setcounter{secnumdepth}{0}
Time: 20 minutes (5 minutes prep, 15 minutes cooking)
Serves: 8 (about 3 cups of sauce)

\begin{multicols}{2}
\subsection*{Ingredients}
\begin{itemize}
    \item 2 cups ketchup (see \nameref{ketchup})
    \item 1 cup water
    \item \( \frac{1}{4} \) cup white vinegar
    \item \( \frac{1}{4} \) cup white vinegar
    \item \( \frac{1}{4} \) cup brown sugar
    \item 2 Tablespoons Worcestershrie sauce
    \item 1 Tablespoon chile powder
    \item 1 Tablespoon cumin
    \item 1 teaspoon kosher salt
    \item 1 teaspoon black pepper
    \item hot sauce to taste

\end{itemize}

\subsection*{Hardware}
\begin{itemize}
    \item Sauce pan
\end{itemize}
\clearpage

\subsection*{Instructions}
\begin{enumerate}
    \item Mix all ingredients in sauce pan.
    \item Set over medium-low heat for 10-15 minutes, until warm an incorporated.

\end{enumerate}

\subsection*{Notes}
\begin{itemize}
    \item Based on Aaron Franklin's recipe, as seen in Franklin Barbecue: A Meat-Smoking Manifesto.
    \begin{itemize}
        \item main differences are addition of hot sauce, and change of most ingredient ratios. The flavor is similar, though this is significantly easier to measure/make, due to the new ratios.
    \end{itemize}
    \item Store in the fridge for upwards of a month.
\end{itemize}
\end{multicols}
\clearpage