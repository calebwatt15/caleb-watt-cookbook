\section{Pie Crust}
\label{pieCrust}
\setcounter{secnumdepth}{0}
Time: 1 hour (30 minutes prep, 30 minutes inactive)
Serves: 1 pie crust

\begin{multicols}{2}
\subsection*{Ingredients}
\begin{itemize}
    \item 2.5 ounces butter
    \item 1.5 ounces lard (or shortening, if you must)
    \item 6 ounces all-purpose flour
    \item \( \frac{1}{2} \) teaspoon table salt
    \item \( \frac{1}{2} \) cup ice water
    \item Additional flour for rolling dough
\end{itemize}

\subsection*{Hardware}
\begin{itemize}
    \item Medium mixing bowl
    \item Small bowl
    \item Rolling pin
    \item Quart-sized ziploc bag
\end{itemize}
\clearpage

\subsection*{Instructions}
\begin{enumerate}
    \item Cut lard into small pieces, about the size of 2-3 peas put together.
    \item Place lard on small plate in the freezer while working on butter.
    \item Chop butter into small pieces (maybe \( \frac{1}{6} \) tablespoon squares, about as big as 2-3 peas combined).
    \item Leave in freezer for at least 10 minutes while doing the next steps.
    \item Sift flour and salt into mixing bowl.
    \item Run hands under cold water to drop temperature while working on the fats.
    \item Drop butter parts into dry-mix (placing lard back into freezer immediately). Using fingers (or forks or pastry wire thingies, if you insist) start crushing butter into the dry mix.
    \item Eventually you’ll have the butter in tiny parts and flakes throughout the dry mix.
    \item Add lard pieces to the mixture, and crush it into the flour mix in the same way as the butter.
    \item Take small bowl of ice-cold water, use your hand to scoop a small amount, sprinkle it around the dry-mix.
    \item Use your fingers to mix the dry-mix around and soak in the small amount of water. Continue doing this until the pie dough holds together, but is not super sticky.
    \item Once the dough holds together, roll it all into a ball, then flatten the ball slightly with your palm.
    \item Add a small amount of flour on both sides of the dough, then store in ziploc bag.
    \item Place bag in fridge for 30 minutes to rest and cool down.
    \item Remove dough from fridge, add small amount of flour on both sides.
    \item Add flour to flat surface or tea towel for rolling out. Flour rolling pin.
    \item Set dough down, and flatten it out. The crust should end up less than \( \frac{1}{2} \) inch thick, and approximately round in shape. It needs to be big enough to cover the entire bottom and sides of a pie pan.
    \item Move pie crust from flat surface to pie pan for baking the pie.

\end{enumerate}

\subsection*{Notes}
\begin{itemize}
    \item Based on Alton Brown’s recipe: https://www.foodnetwork.com/recipes/alton-brown/pie-crust-recipe-1915025 
    \begin{itemize}
        \item Main differences are that I mix it all by hand (no need for a food processor) and I do not blind-bake the crust (though you could, should be fine.)
        \item Additionally I currently use a different ratio of butter to lard (this changes occasionally, as I try new ratios.)
    \end{itemize}
    \item You can make larger recipes for cobblers, top crusts, even dumplings. For Chicken and Dumplings, roll the crust to about \( \frac{1}{4} \) inch thick, then cut into rectangles and place in soup.
    \item The amount of ice water that gets added varies depending on moisture in the air and temperature.
    \item Working in a cooler kitchen is ideal, as you don’t want the fats to melt before baking.
    \item You don’t have to use butter and lard. The amount of butter or lard can also vary. This is my favorite ratio so far.
    \item Pie crust gets its flaky texture from the fat flakes within the crust melting away during the baking process. While most fats can do another fat’s job, in the case of pie crust, you want fats that are solid when the crust is rolled out. Butter, lard, and shortening all work well. Any sort of oil would not really work. No olive oil for your health or whatever. If you want health food, don’t eat pie.
    \item You can pre-bake the crust a bit in the oven, in order to keep it from getting too soggy. This may be required for some pie mixtures, but I usually don’t using this recipe.
\end{itemize}
\end{multicols}
\clearpage