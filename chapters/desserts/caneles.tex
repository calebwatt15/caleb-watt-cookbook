\section{Canelés}
\label{caneles}
\setcounter{secnumdepth}{0}
Time: 1 hour 42 minutes minimum (30 minutes prep, 1-48 hours sitting, 1 hour 12 minutes baking)
Serves: 8

\begin{multicols}{2}
\subsection*{Ingredients}
\begin{itemize}
    \item 500 grams whole milk
    \item 50 grams butter
    \item 1 vanilla bean, split open and insides scraped out (or 14 grams vanilla extract)
    \item 100 grams all purpose flour
    \item 250 grams powdered sugar
    \item 2 whole eggs
    \item 2 egg yolks
    \item 14 grams spiced dark rum
    \item Extra butter to healivy line moulds
\end{itemize}

\subsection*{Hardware}
\begin{itemize}
    \item Sauce pan
    \item Large mixing bowl
    \item Whisk
    \item Baking sheet
    \item Canelé moulds (ideally copper)
\end{itemize}
\clearpage

\subsection*{Instructions}
\begin{enumerate}
    \item Combine 500 grams milk, split vanilla bean and insides, and 50 grams butter in a sauce pan. Do not add vanilla extract here if using.
    \item Bring the milk to a light boil for 2 minutes and ensure the butter is melted.
    \item Remove milk from heat.
    \item Whisk 100 grams flour and 250 grams powdered sugar in a mixing bowl.
    \item Add 2 eggs and 2 egg yolks to the dry ingredients and stir just to combine.
    \item Remove the bean from the milk and discard the bean husk.
    \item Add a small amount of hot milk to the flour/egg mixture in the bowl. Stir to combine.
    \item Continue to drizzle in hot milk slowly while stiring to combine. Add milk to different areas of the mixture to prevent the bowl from getting too hot, or any of the eggs from getting too hot and scrambling.
    \item Add 14 grams spiced dark rum. Add vanilla extract here if using instead of the bean.
    \item Cover the bowl, place in the fridge, and let sit for minimum one hour. These do even better after having sat for 1-2 days in the fridge.
    \item Work the mixture with a whisk until it is all homogenous and liquid again before pouring and using.
    \item Place a clean baking sheet at the bottom rack and preheat oven to 465F.
    \item Heavily coat the inside of the moulds with butter.
    \item Fill the moulds to about 80\% full.
    \item Quickly place the moulds on the baking sheet.
    \item Do not open the oven until all baking is done.
    \item Leave the canelés baking for 12 minutes at 465F.
    \item Reduce the heat to 355F and bake for an additional hour.
    \item Remove from oven after the hour, the canelés should have a dark, rich caramel color on the outside.
    \item Remove from the mould while still hot.
    \item Enjoy.
\end{enumerate}

\subsection*{Notes}
\begin{itemize}
    \item These were shown to me by Marion Bidron, a friend.
    \begin{itemize}
        \item Main difference is I used a whole bean, rather than extract, when I can get one, and I bought copper moulds, so my temps and times are a little different (see below note).
        \item I got some notes on temperatures for copper, as well as the use of rum from this recipe: \url{https://www.marmiton.org/recettes/recette_canneles-bordelais_11439.aspx}
    \end{itemize}
    \item While copper moulds are more traditional, and will lead to darker, chewier crusts, silicon moulds are much more commonly found in the states. I cannot recommend a proper copper mould enough. The canelés come out with such a better outside. If you use silicon, bake the canelés at 430F for 20 minutes, rather than 465F for 12.
    \item These will have a chewy, dark outside, and a super soft, custardy inside.
    \item While you can use vanilla extract, it's not as rich. I splurge on real beans when I can.
\end{itemize}
\end{multicols}
\clearpage