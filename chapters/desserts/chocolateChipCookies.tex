\section{Chocolate Chip Cookies}
\label{chocolateChipCookies}
\setcounter{secnumdepth}{0}
Time: 1 day (15 minutes prep, overnight resting, 10 minutes baking)
Serves: About 30 cookies

\begin{multicols}{2}
\subsection*{Ingredients}
\begin{itemize}
    \item 8 ounces non-salted butter, at room temperature
    \item 2 eggs
    \item 5 ounces granulated white sugar
    \item 5.75 ounces dark brown sugar
    \item 1 teaspoon vanilla extract
    \item 11 ounces all purpose flour
    \item 1 scant teaspoon table salt
    \item 1 teaspoon baking soda
    \item 12 ounces semisweet chocolate chips
    \item 4 ounces nuts (optional, recommend walnuts or pecans)
\end{itemize}

\subsection*{Hardware}
\begin{itemize}
    \item Large mixing bowl
    \item parchment paper
    \item cookie sheet
    \item flexible spatular
    \item wire cooling rack
\end{itemize}
\clearpage

\subsection*{Instructions}
\begin{enumerate}
    \item Combine and mix 8 ounces butter, 2 eggs, 5 ounces white sugar, and 5.75 ounces dark brown sugar until smooth.
    \item Mix in 1 teaspoon vanilla extract.
    \item Add 11 ounces flour, 1 scant teaspoon salt, 1 teaspoon baking soda.
    \item Mix until well combined, but mix as little as possible.
    \item Add 12 ounces chocolate chips and 4 ounces nuts if desired. Cut into the dough with as little work as possible to prevent additional gluten formation.
    \item Cover with parchment paper and leave in the fridge overnight to cool down, rest, and allowing the fats to re-solidify.
    \item Preheat oven to 375F.
    \item Remove dough when about ready to bake.
    \item Grab chunks of dough, about 1 Tablespoon in size.
    \item Roll into a ball, then flatten slightly.
    \item Place disk on ungreased cookie sheet.
    \item Repeat as above, keeping cookies with about 1-2 inches between them (as they will spread).
    \item Once your cookie sheet is full, place in the 375F oven to bake.
    \item Bake for 9-10 minutes, until cookies are just barely crispy on the edge, but still a little raw looking in the middle.
    \item Remove cookie sheet from the oven and allow to sit for 2 minutes undisturbed. Cookies will bake slightly more at this time.
    \item Carefully remove cookies to a wire cooling rack and let cook for another 2-5 minutes.
    \item Consume with milk.
\end{enumerate}

\subsection*{Notes}
\begin{itemize}
    \item Based on the original (well, the original modern version) Nestle Toll House Chocolate Chip Cookie recipe: \url{https://cooking.nytimes.com/recipes/1019232-toll-house-chocolate-chip-cookies}.
    \begin{itemize}
        \item This recipe was licensed by Nestle from Ruth Wakefield.
        \item There is rumor that Ruth Wakefield's recipe used 14 ounces of chocolate, rather than 12. I can't comment on that, as I don't have any records of her recipe.
        \item Main differences are slightly different ingredient amounts (less flour, more brown sugar, less white sugar), change brown sugar to dark brown sugar, resting overnight, more instructions on how to form and bake the cookies, and the change from volumetric measurements to weighted measurements. While I know doing baking by weight is more European than American, and these are as American in origin as you can get, baking by volume allows for a more consistent baking experience.
    \end{itemize}
    \item There is what seems to be about a 15 second window that is the perfect time to pull these out before they are over cooked, but while they will still have a gooey middle after cooking on the sheet for 2 more minutes. Knowing exactly when to pull cookies comes with experience. Just try to keep a consistent size and cookie placement, and get to know your oven really well. Practice makes perfect, and even slightly overbaked or undercooked cookies are delicious.
    \item A note on chocolate. Nestle chocolate chips, while the "original", are frankly not that great. I recommend using the absolute best chocolate you can reasonably find and afford for these, as that takes the cookies to the next level. Ghirardelli are the most commonly found decent chocolate chips, and what my mom recommends. I've heard good things about Trader Joe's chips, though I've not tried them yet. Otherwise I've seen REALLY fancy chocolate, but did not want to spend that much on cookies.
    \item I changed ingredient amounts slightly from the original recipe to get a slightly chewier final cookie. the dark brown sugar's added molasses makes it chewier, as does using more brown than white sugar. The resting voernight allows the fats to solidify, so that cookies spread a little less while baking, which makes them less crispy overall.
    \item There is a Good Eats episode with a different cookie recipe, Alton Brown gets a little more into the ratios and chemistry for chewy versus non-chewy cookies. Recommend watching it if you can find it. I believe it was "Three Chips for SIsterm Marsha", season 3, episode 6.
    \item If you want a crispier cookie (you heretic), add white sugar, reduce brown sugar, and bake from room temperature.
    \item I grew up eating these cookies made by my mom. Until 2018 I thought they were some dark secret family recipe. Turns out she just used the Nestle Toll house recipe our whole life... I felt a little betrayed by that.
\end{itemize}
\end{multicols}
\clearpage