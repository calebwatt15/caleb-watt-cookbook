\section{Meringues}
\label{meringues}
\setcounter{secnumdepth}{0}
Time: 1 hour 15 minutes (15 minutes prep, 1 hour baking)
Serves: 12

\begin{multicols}{2}
\subsection*{Ingredients}
\begin{itemize}
    \item 3 egg whites
    \item Pinch of salt
    \item 3 \( \frac{1}{2} \) ounces \nameref{vanillaSugar} (98g)
    \item 3 ounces raw cane sugar (90g)
\end{itemize}

\subsection*{Hardware}
\begin{itemize}
    \item Fast stand mixer
    \item small mixing bowl
    \item Two spoons
    \item 1 large cookie sheet
    \item Parchment paper to cover cookie sheet
\end{itemize}
\clearpage

\subsection*{Instructions}
\begin{enumerate}
    \item Place 3 eggs in bowl of stand mixer.
    \item Add a pinch of salt.
    \item Whipe the egg whites until they are very stiff.
    \item Combine 3 \( \frac{1}{2} \) ounces vanilla sugar and 3 ounces raw cane sugar in a small bowl.
    \item As the mixer is mixing the egg whites, add in about \( \frac{1}{3} \) of the sugar mixture.
    \item Allow the sugar to fully incorporate before adding the next \( \frac{1}{3} \) of it.
    \item Do this again for the final part of the sugar.
    \item Make sure to cover your cookie sheet with parchment paper.
    \item Pre-heat the oven to 240F.
    \item Once all the sugar is fully incorporated, take two spoons and scoop about one heaping tablespoon on one.
    \item Use the second spoon to scoop the meringue onto the parchment paper.
    \item While the meringue can be placed pretty close together, they will expand slightly, and should not touch.
    \item Place the meringues in the middle of the pre-heated oven and allow to bake for 55 minutes without opening the oven.
    \item Check the meringue after 55 minutes, if they have just barely gotten a kiss of brown on top, they are probably done.
    \item Allow the meringues to cool for 10-15 minutes before eating. They should be slightly crispy throughout, but collapse when eaten. They should not remain gooey.
\end{enumerate}

\subsection*{Notes}
\begin{itemize}
    \item I got this recipe from my friend, Nicolas Bidron, who adapted from his mtoher's recipe.
    \begin{itemize}
        \item The main difference is the conversion to ounces (which matches the rest of the cookbook), as well as using 98g of vanilla sugar, whereas Nicolas uses equal parts granulated and raw sugar, and then uses a single 8-gram envelope of "sucre vanillé", or vanilla sugar. These envelopes also have a little extra starch sometimes, but I find it's not necessary.
    \end{itemize}
    \item You can make larger meringues, but they will require additional cooking time and will brown further.
    \item The easiest way to learn the exact cooking times is to just make a consistent size and practice with your oven.
    \item While they will not last long (as they obsorb moisture from the air and begin to fall apart after a couple days), they can be stored in an air-tight container for a week or so.
    \item Similar technique (whipping air into egg whites) can be used to make \nameref{floatingIslands} or a meringue pie, though the cooking tehcnique varies significantly.
\end{itemize}
\end{multicols}
\clearpage