\section{Crème Pâtissière}
\label{cremePatissiere}
\setcounter{secnumdepth}{0}
Time: 30 minutes (10 minutes prep, 20 minutes cooking)
Serves: about 2 \( \frac{1}{2} \) cups

\begin{multicols}{2}
\subsection*{Ingredients}
\begin{itemize}
    \item 2 cups whole milk
    \item 5 egg yolks
    \item 6 \( \frac{1}{2} \) ounces granulated sugar
    \item 2 \( \frac{1}{2} \) ounces all purpose flour
    \item 1 vanilla bean (substitute with 1 \( \frac{1}{2} \) Tablespoons vanilla extract)
    \item \( \frac{1}{2} \) ounces unsalted butter
\end{itemize}

\subsection*{Hardware}
\begin{itemize}
    \item Small pot
    \item Stand mixer
    \item Stock pot
    \item Whisk
    \item Plastic wrap
\end{itemize}
\clearpage

\subsection*{Instructions}
\begin{enumerate}
    \item Put two cups of milk into small pot.
    \item Allow milk to begin to warm, you want it to reach a very slight boil while doing the next few steps.
    \item Place 5 egg yolks in a stand mixer, turn it on medium.
    \item Gradually add in 6 \( \frac{1}{2} \) ounces sugar, ensuring first part is mixed in before adding more.
    \item Continue to beat this until it is pale yellow and forms ribbons on top if you drip some mixture on top of the rest of the mixture (about 2-3 minutes)
    \item Continue beating mixture while adding in 2 \( \frac{1}{2} \) ounces flour.
    \item Remove milk from heat.
    \item Split a vanilla bean down the middle, long ways.
    \item Scrape the inside of the bean into the egg mixture, then throw in the empty shell.
    \item Gradually dribble milk into the egg mixture while mixing.
    \item The goal is to bring the eggs up to temperature slowly until all milk is mixed in.
    \item Remove the bean shell.
    \item Pour the mixture into the stock pot over medium-low heat.
    \item At this point, you must constantly mix and scrape all sides and bottom of the dish while it comes to temperature.
    \item After 15 or more minutes of constant low heat the mixture will begin to thicken.
    \item Remove when it is at the desired thickness (it will thicken a little more as it cools, but adding the butter thins it a tad.)
    \item Add \( \frac{1}{2} \) ounces unsalted butter and stir to combine.
    \item If you use vanilla extract instead of a bean, add it at this point, otherwise too much may evaporate while cooking.
    \item Cover with plastic wrap, ensuring that the custard is completely covered and the plastic wrap is pressed lightly onto the surface.
    \item Allow to cool to room temperature before storing in the fridge.
\end{enumerate}

\subsection*{Notes}
\begin{itemize}
    \item This is based on the recipe of Julia Child, Simone Beck, and Louisette Bertholle, as seen in Mastering the Art of French Cooking, Volume 1, page 590. 
    \begin{itemize}
        \item Main differences are that I use a vanilla bean over vanilla extract (when I can find them at a reasonable price), and I don't usually substitute vanilla for other flavors (such as cognac or kirshwasser).
        \item I converted units from volume to weight to get a more consistant cooking experience.
    \end{itemize}
    \item "Crème Pâtissière" translates to "Pastry Cream", and is a custard used in many French pastry recipes (mille-fuielles, tart au fraises, etc).
    \item If you are using this custard a pudding-type dessert, such as for \nameref{bananaPudding}, then make sure to take it off the heat before it is too thick.
    \item If you need to stiffen it a little more for something like mille-fuielles, then you can add in a half tablespoon of plain geletin.
\end{itemize}
\end{multicols}
\clearpage