\section{Pan de Polvo}
\label{panDePolvo}
\setcounter{secnumdepth}{0}
Time: 2 days (1 hour prep, 1 hour baking (in batches), 2 days resting)
Serves: About 80 cookies (they are small)

\begin{multicols}{2}
\subsection*{Ingredients}
\begin{itemize}
    \item 2 cups water
    \item 6 cinnamon sticks
    \item 4 Tablespoons anise seeds
    \item 1 pound lard
    \item 2 pounds flour
    \item 3 cups sugar
    \item \( \frac{1}{4} \) cup ground cinnamon
\end{itemize}

\subsection*{Hardware}
\begin{itemize}
    \item Sauce pan
    \item Cheese cloth and butcher's twine, or, alternatively, a large tea strainer
    \item Large mixing bowl
    \item Three cookie sheets
    \item Cooling rack
    \item Sharp Knife
\end{itemize}
\clearpage

\subsection*{Instructions}
\begin{enumerate}
    \item Bring 2 cups water to a light boil in the sauce pan.
    \item Combine 6 cinnamon sticks and 4 tablespoons anise seeds in cheese cloth.
    \item Tie packet with butcher's twine to keep the seeds from falling out.
    \item Place anise packet into boiling water for about 15 minute to create anise tea.
    \item Cut 1 pound lard into 2 pounds of flour.
    \item Add \( \frac{1}{4} \) cup of anise tea into the dough.
    \item Knead the dough until it all comes together and becomes less sticky, as well as very smooth.
    \item Preheat oven to 375F.
    \item Take a handful of the dough, and create a little log with a 1 inch diameter.
    \item Take your knife and cut the log every \( \frac{1}{4} \) inch to create small disks.
    \item Place disks on cookie sheet until it is full, but not too crowded.
    \item Allow the first sheet to start baking for 20 minutes.
    \item Cookies should bake until they are just slightly browned, but baked through.
    \item While one sheet is baking, finish filling out the other two sheets.
    \item Combine 3 cups of sugar and \( \frac{1}{4} \) cup ground cinnamon in a large bowl.
    \item Place the next sheet of cookies in the oven to bake for 20 minutes while doing the next steps.
    \item Take hot cookies from the oven and place each cookie into the cinnamon sugar.
    \item Toss cinnamon sugar over each cookie to coat, then place on cooling rack.
    \item Continue the above steps until all cookies are baked and coated in cinnamon sugar.
    \item Allow cookies to sit out uncovered overnight, then let them sit for at least 48 hours total (covered after the first night) before consuming.
\end{enumerate}

\subsection*{Notes}
\begin{itemize}
    \item Based on Genius Kitchen recipe: \url{http://www.geniuskitchen.com/recipe/pan-de-polvo-mexican-shortbread-27623}.
    \begin{itemize}
        \item Main differences are that I use lard, rather than shortening, and I force the cookies to sit for 2 days.
    \end{itemize}
    \item This recipe is huge, it should be possible to scale it down, but I've not tried.
    \item These cookies really should rest for at least a day or two. If consumed fresh from the oven, the lard has not cooled properly and they don't crumble and melt properly in the mouth.
    \item I use lard rather than shortening. I don't find the taste of lard to impede the recipe, and it is more traditional than shortening.
    \item These are a shortbread recipe, essentially. The unique flavor comes from the anise tea used to help flavor the dough.
\end{itemize}
\end{multicols}
\clearpage